\documentclass[a4paper,10pt]{article}
\usepackage{amsthm}
\usepackage{amsmath}
\usepackage{amsfonts}
\usepackage{amssymb}
\usepackage{mathtools}
\usepackage{listings}
\usepackage{fullpage}
\usepackage{color}
\title{PX275 Moodle Test Questions}
\author{Joe Tilley}
\begin{document}
\maketitle
\section{Test 1}
\subsection{Application to Electromagnetism}
\subsubsection{Question Variables}
\begin{lstlisting}
powerdisp:true;

/* Functions */
Grad(f) := diff(f,x)*i + diff(f,y)*j + diff(f,z)*k;
dot(f,g) := ev(coeff(f,i)*coeff(g,i)+coeff(f,j)*coeff(g,j)+coeff(f,k)*coeff(g,k),expand,simp);
mag(v) := sqrt(coeff(v,i)^2 + coeff(v,j)^2 + coeff(v,k)^2);
DirDeriv(f,r) := dot(Grad(f),r)/mag(r);

/* Variables */
phi: Q/(4*pi*epsilon[0]) * e^(-z)/(x^2+y^2);
point: [0,0,0];
point[1]: rand_with_prohib(-5,5,[0]);     /* No 0 to prevent gradient of phi =0*/
point[2]: rand_with_prohib(-5,5,[0,-point[1]]);     /* No -point[1] to prevent c=0 */
point[3]: rand(11)-5;
dir1: rand([i,j,k]);
dir2: rand_with_prohib(-3,3,[0])*i+rand_with_prohib(-3,3,[0])*j+rand_with_prohib(-3,3,[0])*k;
dir3: i+j+c*k;

/* Teacher Answers */
ta1: factor(Grad(phi));
ta2: subst(point[3],z,subst(point[2],y,subst(point[1],x,DirDeriv(phi,dir1))));
ta3: subst(point[3],z,subst(point[2],y,subst(point[1],x,DirDeriv(phi,dir2))));
soln: solve(subst(point[3],z,subst(point[2],y,subst(point[1],x,DirDeriv(phi,dir3))))=0,c)
ta4: rhs(soln[1]);     /* Accesses value of c from the oddly saved form that 'solve' saves as */
\end{lstlisting}
\subsubsection{Question Text}
Electric potential, \(\phi\), is a scalar quantity. This causes an electric field, which is a vector quantity, related to the potential by the expression \(\mathbf{E} = -\nabla \phi\). A student arranges some electrically-charged plates, creating an electric potential given by \[ \phi(x,y,z) = \frac{Q}{4 \pi \varepsilon_{0}} \cdot \frac{e^{-z}}{x^2+y^2} \] where \(Q\) represents the charge on the plates.

Note that \(\pi\), $e$, $\varepsilon_{0}$ and $\sqrt{n}$ can be typed as pi, and e, and epsilon[0], and sqrt(n)in the answer box. You should use the suggested form for your solution when it is provided. You must give your answer in exact form. Factorisation is permitted in your answers, if you choose to do so.

(a) Find \(\nabla \phi\).

\(\nabla \phi = \)[[input:ans1]][[validation:ans1]][[feedback:prt1]]

(b) Calculate the directional derivatives of $\phi$ along the direction \(\mathbf u\), denoted by \(\nabla_{\textbf{u}} \phi\), at the position (@point[1]@,@point[2]@,@point[3]@) for the following choices of \(\mathbf u\).

(i) \(\mathbf{u} = \mathbf{@dir1@} \)

\(\nabla_{\textbf{u}} \phi =\) [[input:ans2]][[validation:ans2]][[feedback:prt2]]

(ii) \(\mathbf{u} = @coeff(dir2,i)@ \, \mathbf{i} + @coeff(dir2,j)@ \, \mathbf{j} +@coeff(dir2,k)@ \, \mathbf{k}\)

\(\nabla_{\textbf{u}} \phi =\) [[input:ans3]][[validation:ans3]][[feedback:prt3]]

(c) At this position, the directional derivative is zero along direction \(\mathbf{u} = \mathbf{i} + \mathbf{j} + c \, \mathbf{k}\). Determine the value of \(c\).

\(c\) = [[input:ans4]][[validation:ans4]][[feedback:prt4]]
\subsubsection{Hint}
Remember that the directional derivative of \(\phi\) at \((x,y,z)\) in direction \(\mathbf{u}\) is given by \begin{align*} \nabla_{\textbf{u}}\phi(x,y,z) = \nabla\phi(x,y,z) \cdot \frac{\textbf{u}}{|\textbf{u}|} \end{align*}
\subsubsection{Model Solution}
(a) We calculate
\begin{align*}
\nabla\phi &= \frac{\partial \phi}{\partial x} \, \mathbf i + \frac{\partial \phi}{\partial y} \, \mathbf j + \frac{\partial \phi}{\partial z} \, \mathbf k \\
&= \left(@diff(phi,x)@\right) \mathbf i + \left(@diff(phi,y)@ \right) \mathbf j + \left(@diff(phi,z)@\right) \mathbf k \\
&= \frac{-e^{-z}Q}{4 \pi \varepsilon_{0} (x^2+y^2)^2} \left( 2x \, \mathbf i + 2y \, \mathbf j + \left(x^2+y^2\right) \mathbf k \right) \end{align*}
(b) Recall the formula \begin{align*} \nabla_{\textbf{u}}\phi(x,y,z) = \nabla\phi(x,y,z) \cdot \frac{\textbf{u}}{|\textbf{u}|} \end{align*} So, evaluating \(\phi\) at \( (x,y,z) = \left(@point[1]@, @point[2]@, @point[3]@\right)\) gives us
\begin{align*}
\nabla \phi \left(@point[1]@, @point[2]@, @point[3]@\right) &= @(-e^(-point[3])*Q)/(4*(point[1]^2+point[2]^2)^2 *pi*epsilon[0])@ \left( @2*point[1]@ \, \mathbf i + @2*point[2]@ \, \mathbf j + @point[1]^2+point[2]^2@ \, \mathbf k \right)
\end{align*}
Hence, we can proceed to calculate the directional derivatives by taking the dot product of this vector with the normalised vector of \(\textbf{u}\).(i) Since \(\textbf{u}\) is already a unit vector, it has magnitude \(1\), so we can begin our calculation with \(\textbf{u}\), as this is equal to \(\frac{\textbf{u}}{|\textbf{u}|}\).
\begin{align*} \nabla_{\textbf{u}}\phi\left(@point[1]@, @point[2]@, @point[3]@\right) &= @(-e^(-point[3])*Q)/(4*(point[1]^2+point[2]^2)^2 *pi*epsilon[0])@ \left( @2*point[1]@ \, \mathbf i + @2*point[2]@ \, \mathbf j + @point[1]^2+point[2]^2@ \, \mathbf k \right) \cdot \left(\textbf{@dir1@} \right) \\ &= @(-e^(-point[3])*Q)/(4*(point[1]^2+point[2]^2)^2 *pi*epsilon[0])@ \left( @2*point[1]@\cdot@coeff(dir1,i)@ + @2*point[2]@\cdot@coeff(dir1,j)@ + @point[1]^2+point[2]^2@ \cdot @coeff(dir1,k)@ \right) \\ &= @ta2@ \end{align*}
(ii) We first calculate the magnitude of \(\textbf{u}\) \begin{align*} \mathbf u = @coeff(dir2,i)@ \, \mathbf{i} + @coeff(dir2,j)@ \, \mathbf{j} +@coeff(dir2,k)@ \, \mathbf{k} \quad \Rightarrow \quad |\textbf{u}|=\sqrt{(@coeff(dir2,i)@)^2 + (@coeff(dir2,j)@)^2 + (@coeff(dir2,k)@)^2}=@mag(dir2)@ \end{align*} Then, we compute the dot product using the normalised vector \begin{align*} \nabla_{\textbf{u}}\phi\left(@point[1]@, @point[2]@, @point[3]@\right) &= @(-e^(-point[3])*Q)/(4*(point[1]^2+point[2]^2)^2 *pi*epsilon[0])@ \left( @2*point[1]@ \, \textbf{i} +  @2*point[2]@ \, \textbf{j} + @point[1]^2+point[2]^2@ \, \textbf{k} \right) \cdot \frac{1}{@mag(dir2)@} \left(@coeff(dir2,i)@ \, \textbf{i} + @coeff(dir2,j)@ \, \textbf{j} + @coeff(dir2,k)@ \, \textbf{k} \right) \\ &=@(-e^(-point[3])*Q)/(4*(point[1]^2+point[2]^2)^2 *pi*epsilon[0]*mag(dir2))@ \left( @2*point[1]@\cdot@coeff(dir2,i)@ + @2*point[2]@\cdot@coeff(dir2,j)@ + @point[1]^2+point[2]^2@ \cdot @coeff(dir2,k)@ \right) \\ &= @ta3@ \end{align*}
(c) To find \(c\), we simply set up our equation as in the previous parts and solve for \(c\). We first calculate the magnitude of \(\textbf{u}\) \begin{align*} \mathbf u = \mathbf{i} + \mathbf{j} + c \, \mathbf{k} \quad \Rightarrow \quad |\textbf{u}|=\sqrt{(@coeff(dir3,i)@)^2 + (@coeff(dir3,j)@)^2 + (@coeff(dir3,k)@)^2}=@mag(dir3)@ \end{align*} Then, we compute the dot product using the normalised vector, set this equal to zero, and then solve for \(c\).
\begin{gather*} \nabla_{\textbf{u}}\phi\left(@point[1]@, @point[2]@, @point[3]@\right) = 0 \\
\Rightarrow \quad @(-e^(-point[3])*Q)/(4*(point[1]^2+point[2]^2)^2 *pi*epsilon[0])@ \left( @2*point[1]@ \, \textbf{i} +  @2*point[2]@ \, \textbf{j} + @point[1]^2+point[2]^2@ \, \textbf{k} \right) \cdot \frac{1}{@mag(dir3)@} \left( \textbf{i} + \textbf{j} + @coeff(dir3,k)@ \, \textbf{k} \right) = 0 \\ \Rightarrow \quad @(-e^(-point[3])*Q)/(4*(point[1]^2+point[2]^2)^2 *pi*epsilon[0]*mag(dir3))@ \left( @2*point[1]@\cdot@coeff(dir3,i)@ + @2*point[2]@\cdot@coeff(dir3,j)@ + @point[1]^2+point[2]^2@ \cdot @coeff(dir3,k)@ \right) = 0 \\ \Rightarrow \quad @2*point[1]@\cdot@coeff(dir3,i)@ + @2*point[2]@\cdot@coeff(dir3,j)@ + @point[1]^2+point[2]^2@ \cdot @coeff(dir3,k)@ = 0 \\ \Rightarrow \quad c = @ta4@
\end{gather*}
\subsubsection{Question Note}
Find \(\nabla \phi\), and directional derivatives at @point@ in directions @dir1@ and @dir2@, and find \(c\) such that \(\nabla_{(1,1,c)} \phi = 0\).
\subsubsection{Feedback Variables}
\begin{lstlisting}
/* Feedback Variables /*
ecf2: subst(point[3],z,subst(point[2],y,subst(point[1],x,dot(ans1,dir1)/mag(dir1))));     /* The correct method using their answer from part (a) */
nomag: (ta3)*mag(dir2);     /* answer if they forgot to normalise */
ecf3: subst(point[3],z,subst(point[2],y,subst(point[1],x,dot(ans1,dir2)/mag(dir2))));     /* The correct method using their answer from part (a) */

/* Error carried forward calculations */
ecfsoln: solve(subst(point[3],z,subst(point[2],y,subst(point[1],x,dot(ans1,dir3)/mag(dir3))))=0,c);
ecf4: rhs(ecfsoln[1]);     /* Accesses value of c from the oddly saved form that 'solve' saves as */ 
\end{lstlisting}

\subsection{Application to Maxwell Relations}
\subsubsection{Question Variables}
\begin{lstlisting}
ta: [V,S,P,P,T,V];
lc: [v,s,p,p,t,v];
\end{lstlisting}
\subsubsection{Question Text}
In lectures, we showed that from the expression for internal energy, \(\text{d}U = T \, \text{d}S - P \, \text{d}V\), we can obtain the Maxwell relation given by \[\left(\frac{\partial T}{\partial V}\right)_{S} = -\left(\frac{\partial P}{\partial S}\right)_{V}\] In this question, we consider two more of the thermodynamic potentials, but you do not need to know any thermodynamics to work out the answer! Hint: the expressions are exact differentials. Note that the notation $\left(\frac{\partial f}{\partial x}\right)_y$ means "the partial derivative of $f$ with respect to $x$ with $y$ held constant."

(a) Starting with the expression for enthalpy \[ \text{d}H = T \, \text{d}S + V \, \text{d}P \] Complete the corresponding Maxwell relation by giving the symbol which should be in place of each Roman numeral. \[ \left(\frac{\partial T}{\partial P}\right)_{S} = \left(\frac{\partial \color{red}{\textbf{(i)}}}{\partial \color{red}{\textbf{(ii)}}}\right)_{\color{red}{\textbf{(iii)}}} \]
\( \color{red}{\textbf{(i)}} =\) [[input:ans1]][[validation:ans1]]
\( \color{red}{\textbf{(ii)}} =\) [[input:ans2]][[validation:ans2]]
\( \color{red}{\textbf{(iii)}} =\)[[input:ans3]][[validation:ans3]][[feedback:prt1]]

(b) Another thermodynamic potential is the Helmhotz function, given by the expression \[ \text{d}F = -S \, \text{d}T - P \, \text{d}V \] Complete the Maxwell relation by giving the symbol which should be in place of each Roman numeral. \[ \left(\frac{\partial S}{\partial V}\right)_{T} = \left(\frac{\partial \color{blue}{\mathbf{(i)}}}{\partial \color{blue}{\mathbf{(ii)}}}\right)_{\color{blue}{\mathbf{(iii)}}} \]
\( \color{blue}{\mathbf{(i)}} =\) [[input:ans4]][[validation:ans4]]
\( \color{blue}{\mathbf{(ii)}} =\) [[input:ans5]][[validation:ans5]]
\(\color{blue}{\mathbf{(iii)}} =\) [[input:ans6]][[validation:ans6]][[feedback:prt2]]
\subsubsection{Hint}
Remember that an exact differential is one of the form \begin{align*} \text{d}f = \left(\frac{\partial f}{\partial x}\right)_y \text{d}x + \left(\frac{\partial f}{\partial y}\right)_x \text{d}y \end{align*} where the subscript denotes that that variable is held as a constant. You should compare this to the expressions provided, and then compute further derivatives to reach an equality by using the fact that $\frac{\partial^2 f}{\partial x \partial y} = \frac{\partial^2 f}{\partial y \partial x}$.
\subsubsection{Model Solution}
(a) An exact differential is one of the form \begin{align*} \text{d}f = \left(\frac{\partial f}{\partial x}\right)_y \text{d}x + \left(\frac{\partial f}{\partial y}\right)_x \text{d}y \end{align*} Note that the subscript denotes that that variable is held as a constant. Since the expression for enthalpy is an exact differential, we can say \begin{align*} T = \left(\frac{\partial H}{\partial S}\right)_P \qquad \text{and} \qquad V = \left(\frac{\partial H}{\partial P}\right)_S \end{align*} Now, we differentiate each of these with respect to the same variable as that held as a constant. Again, another variable is held as constant when differentiating \begin{align*} \left(\frac{\partial T}{\partial P}\right)_S = \frac{\partial^2 H}{\partial P \partial S} \qquad \text{and} \qquad \left(\frac{\partial V}{\partial S}\right)_P = \frac{\partial^2 H}{\partial S \partial P} \end{align*} Since $ \frac{\partial^2 H}{\partial P \partial S} = \frac{\partial^2 H}{\partial S \partial P} $ , we get \begin{align*} \left(\frac{\partial T}{\partial P}\right)_S = \left(\frac{\partial V}{\partial S}\right)_P \end{align*} So, $\color{red}{\textbf{(i)}} = V$, $\color{red}{\textbf{(ii)}} = S$ and $\color{red}{\textbf{(iii)}} = P$.

(b) Again, as we have an exact differential, we can follow the general formula to say \begin{align*} -S = \left(\frac{\partial F}{\partial T}\right)_V \qquad \text{and} \qquad -P = \left(\frac{\partial F}{\partial V}\right)_T \end{align*} Now, we differentiate each of these with respect to the same variable as that held as a constant. Again, another variable is held as constant when differentiating \begin{align*} -\left(\frac{\partial S}{\partial V}\right)_T = \frac{\partial^2 F}{\partial V \partial T} \qquad \text{and} \qquad -\left(\frac{\partial P}{\partial T}\right)_V = \frac{\partial^2 F}{\partial T \partial V} \end{align*} Since $ \frac{\partial^2 F}{\partial V \partial T} = \frac{\partial^2 F}{\partial T \partial V} $ , we get \begin{align*} \left(\frac{\partial S}{\partial V}\right)_T = \left(\frac{\partial P}{\partial T}\right)_V \end{align*} So, $\color{blue}{\mathbf{(i)}} = P$, $\color{blue}{\mathbf{(ii)}} = T$ and $\color{blue}{\mathbf{(iii)}} = V$.
\subsubsection{Question Note}
Find the Maxwell relations for the expression for enthalpy and the Helmhotz function.
\subsection{Chain Rule with Partial Derivatives and Polar Co-ordinates 1}
\subsubsection{Question Variables}
\begin{lstlisting}
powerdisp:true;

/* Coefficients */
Coeffs:[0,0];
Coeffs[1]:rand_with_prohib(-5,5,[0]);
Coeffs[2]:rand_with_prohib(-5,5,[0]);

/* Polar co-ordinates */
xpolar: rho*cos(theta);
ypolar: rho*sin(theta);
fpolar: Coeffs[1]*xpolar*exp(Coeffs[2]*ypolar);
f: Coeffs[1]*x*exp(Coeffs[2]*y);     /* f in terms of x and y */

/* Derivatives in x and y AND in polar form */
dfdx: diff(f,x);
dfdxpolar: subst(rho*cos(theta),x,subst(rho*sin(theta),y,dfdx));
dxdt: diff(xpolar,theta);
dxdp: diff(xpolar,rho);
dfdy: diff(f,y)
dfdypolar: subst(rho*cos(theta),x,subst(rho*sin(theta),y,dfdy));
dydt: diff(ypolar,theta);
dydp: diff(ypolar,rho);

/* Teacher Answers */
ta1: factor(diff(fpolar,theta));
ta2: factor(diff(fpolar,rho));
\end{lstlisting}
\subsubsection{Question Text}
Consider the function \[f(x,y) = @f@\]Consider plane polar coordinates, where, as in lectures \begin{align*} x&=\rho\cos(\vartheta) \\ y&=\rho\sin(\vartheta) \\ \rho^2&=x^2+y^2 \end{align*}
(a) Using the chain rule, find \( \frac{\partial f}{\partial \vartheta} \). You must give your answer in polar co-ordinates, i.e. in terms of \(\vartheta\) and $\rho$ and mathematical functions only. Note that $e^{x}$, \(\vartheta\) and \(\rho\) can be typed as exp(x), theta and rho in the answer box. Factorisation is permitted in your answers, if you choose to do so, though you do not need to give your answer in its simplest form.

\( \frac{\partial f}{\partial \vartheta} = \) [[input:ans1]][[validation:ans1]][[feedback:prt1]]

(b) Using the chain rule, find \( \frac{\partial f}{\partial \rho} \). You must give your answer in polar co-ordinates, i.e. in terms of \(\vartheta\) and $\rho$ and mathematical functions only. Note that \(\vartheta\) and $\rho$ can be typed as theta and rho in the answer box. Factorisation is permitted in your answers, if you choose to do so, though you do not need to give your answer in its simplest form.

\( \frac{\partial f}{\partial \rho} = \) [[input:ans2]][[validation:ans2]][[feedback:prt2]]
\subsubsection{Hint}
It may be useful to rewrite $\frac{\partial f}{\partial \vartheta}$ and $\frac{\partial f}{\partial \rho}$ as \[ \frac{\partial f}{\partial \vartheta} = \frac{\partial f}{\partial x}\cdot\frac{\partial x}{\partial \vartheta} + \frac{\partial f}{\partial y}\cdot\frac{\partial y}{\partial \vartheta} \qquad \text{and} \qquad \frac{\partial f}{\partial \rho} = \frac{\partial f}{\partial x}\cdot\frac{\partial x}{\partial \rho} + \frac{\partial f}{\partial y}\cdot\frac{\partial y}{\partial \rho} \] You can find the each of the expressions on the right-hand side to find your solutions.
\subsubsection{Model Solution}
(a) Recall the formula \[ \frac{\partial f}{\partial \vartheta} = \frac{\partial f}{\partial x}\cdot\frac{\partial x}{\partial \vartheta} + \frac{\partial f}{\partial y}\cdot\frac{\partial y}{\partial \vartheta} \] Hence, we find each of the derivatives on the right-hand side of this equation. Using the equations given for plane polar co-ordinates, we find \begin{align*} \frac{\partial f}{\partial x} &= @dfdx@ & \frac{\partial x}{\partial \vartheta} &= @dxdt@ \\ \frac{\partial f}{\partial y} &= @dfdy@ & \frac{\partial y}{\partial \vartheta} &= @dydt@ \end{align*} All that remains is to substitute these expressions in, simplify our solution and convert to polar co-ordinates \begin{align*} \frac{\partial f}{\partial \vartheta} &= \frac{\partial f}{\partial x} \cdot \frac{\partial x}{\partial \vartheta} + \frac{\partial f}{\partial y} \cdot \frac{\partial y}{\partial \vartheta} \\ &= \left(@dfdx@\right)\cdot\left(@dxdt@\right) + \left(@dfdy@\right)\cdot\left(@dydt@\right) \\ &= @factor(dfdx*dxdt+dfdy*dydt)@ \\ &= @ta1@ \end{align*} (b) We perform a similar process. Recall \[ \frac{\partial f}{\partial \rho} = \frac{\partial f}{\partial x}\cdot\frac{\partial x}{\partial \rho} + \frac{\partial f}{\partial y}\cdot\frac{\partial y}{\partial \rho} \] Again, we find each of the derivatives on the right-hand side of this equation. Using the equations given for plane polar co-ordinates, we find \begin{align*} \frac{\partial f}{\partial x} &= @dfdx@ & \frac{\partial x}{\partial \rho} &= @dxdp@ \\ \frac{\partial f}{\partial y} &= @dfdy@ & \frac{\partial y}{\partial \rho} &= @dydp@ \end{align*} All that remains is to substitute these values in, simplify our solution and convert to polar co-ordinates \begin{align*} \frac{\partial f}{\partial \rho} &= \frac{\partial f}{\partial x} \cdot \frac{\partial x}{\partial \rho} + \frac{\partial f}{\partial y} \cdot \frac{\partial y}{\partial \rho} \\ &= \left(@dfdx@\right)\cdot\left(@dxdp@\right) + \left(@dfdy@\right)\cdot\left(@dydp@\right) \\ &=@factor(dfdx*dxdp+dfdy*dydp)@ \\ &= @subst([x=rho*cos(theta), y=rho*sin(theta)],factor(dfdx*dxdp+dfdy*dydp))@ \\ &= @ta2@ \end{align*}
\subsubsection{Question Note}
Find \( \frac{\partial f}{\partial \vartheta} \) and \( \frac{\partial f}{\partial \rho} \) of @f@ in polar coordinates.
\subsubsection{Feedback Variables}
\begin{lstlisting}
/* Feedback Variables */
notpolar1: dfdx*dxdt+dfdy*dydt;
notpolar2: dfdx*dxdp+dfdy*dydp;
\end{lstlisting}

\subsection{Chain Rule with Partial Derivatives and Polar Co-ordinates 2}
\subsubsection{Question Variables}
\begin{lstlisting}
powerdisp:true;

/* Coefficients */
Coeffs:[0,0];
Coeffs[1]: rand([-1,-1/2,1/2,1]);
Coeffs[2]: rand([-1,-1/2,1/2,1]);

/* Polar co-ordinates */
xpolar: rho*cos(theta);
ypolar: rho*sin(theta);
fpolar: k*exp(Coeffs[1]*(xpolar)^2+Coeffs[2]*(ypolar)^2);
f: k*exp(Coeffs[1]*x^2+Coeffs[2]*y^2);     /* f in terms of x and y */

/* Derivatives in x and y AND in polar form */
dfdx: diff(f,x);
dfdxpolar: subst(rho*cos(theta),x,subst(rho*sin(theta),y,dfdx));
dxdt: diff(xpolar,theta);
dxdp: diff(xpolar,rho);
dfdy: diff(f,y)
dfdypolar: subst(rho*cos(theta),x,subst(rho*sin(theta),y,dfdy));
dydt: diff(ypolar,theta);
dydp: diff(ypolar,rho);

/* Teacher Answers */
ta1: factor(diff(fpolar,theta));
ta2: factor(diff(fpolar,rho));
\end{lstlisting}
\subsubsection{Question Text}
Consider the function \[f(x,y) = @f@\]Consider plane polar coordinates, where, as in lectures \begin{align*} x&=\rho\cos(\vartheta) \\ y&=\rho\sin(\vartheta) \\ \rho^2&=x^2+y^2 \end{align*}
(a) Using the chain rule, find \( \frac{\partial f}{\partial \vartheta} \). You must give your answer in polar co-ordinates, i.e. in terms of \(\vartheta\) and $\rho$ and mathematical functions only. Note that $e^{x}$, \(\vartheta\) and \(\rho\) can be typed as exp(x), theta and rho in the answer box Factorisation is permitted in your answers, if you choose to do so, though you do not need to give your answer in its simplest form.

\( \frac{\partial f}{\partial \vartheta} = \) [[input:ans1]][[validation:ans1]][[feedback:prt1]]

(b) Using the chain rule, find \( \frac{\partial f}{\partial \rho} \). You must give your answer in polar co-ordinates, i.e. in terms of \(\vartheta\) and $\rho$ and mathematical functions only. Note that $e^{x}$, \(\vartheta\) and \(\rho\) can be typed as exp(x), theta and rho in the answer box. Factorisation is permitted in your answers, if you choose to do so, though you do not need to give your answer in its simplest form.

\( \frac{\partial f}{\partial \rho} = \) [[input:ans2]][[validation:ans2]][[feedback:prt2]]
\subsubsection{Hint}
It may be useful to rewrite $\frac{\partial f}{\partial \vartheta}$ and $\frac{\partial f}{\partial \rho}$ as \[ \frac{\partial f}{\partial \vartheta} = \frac{\partial f}{\partial x}\cdot\frac{\partial x}{\partial \vartheta} + \frac{\partial f}{\partial y}\cdot\frac{\partial y}{\partial \vartheta} \qquad \text{and} \qquad \frac{\partial f}{\partial \rho} = \frac{\partial f}{\partial x}\cdot\frac{\partial x}{\partial \rho} + \frac{\partial f}{\partial y}\cdot\frac{\partial y}{\partial \rho} \] You can find the each of the expressions on the right-hand side to find your solutions.
\subsubsection{Model Solution}
(a) Recall the formula \[ \frac{\partial f}{\partial \vartheta} = \frac{\partial f}{\partial x}\cdot\frac{\partial x}{\partial \vartheta} + \frac{\partial f}{\partial y}\cdot\frac{\partial y}{\partial \vartheta} \] Hence, we find each of the derivatives on the right-hand side of this equation. Using the equations given for plane polar co-ordinates, we find \begin{align*} \frac{\partial f}{\partial x} &= @dfdx@ & \frac{\partial x}{\partial \vartheta} &= @dxdt@ \\ \frac{\partial f}{\partial y} &= @dfdy@ & \frac{\partial y}{\partial \vartheta} &= @dydt@ \end{align*} All that remains is to substitute these expressions in, simplify our solution and convert to polar co-ordinates \begin{align*} \frac{\partial f}{\partial \vartheta} &= \frac{\partial f}{\partial x} \cdot \frac{\partial x}{\partial \vartheta} + \frac{\partial f}{\partial y} \cdot \frac{\partial y}{\partial \vartheta} \\ &= \left(@dfdx@\right)\cdot\left(@dxdt@\right) + \left(@dfdy@\right)\cdot\left(@dydt@\right) \\ &= @factor(dfdx*dxdt+dfdy*dydt)@ \\ &= @ta1@ \end{align*} (b) We perform a similar process. Recall \[ \frac{\partial f}{\partial \rho} = \frac{\partial f}{\partial x}\cdot\frac{\partial x}{\partial \rho} + \frac{\partial f}{\partial y}\cdot\frac{\partial y}{\partial \rho} \] Again, we find each of the derivatives on the right-hand side of this equation. Using the equations given for plane polar co-ordinates, we find \begin{align*} \frac{\partial f}{\partial x} &= @dfdx@ & \frac{\partial x}{\partial \rho} &= @dxdp@ \\ \frac{\partial f}{\partial y} &= @dfdy@ & \frac{\partial y}{\partial \rho} &= @dydp@ \end{align*} All that remains is to substitute these values in, simplify our solution and convert to polar co-ordinates \begin{align*} \frac{\partial f}{\partial \rho} &= \frac{\partial f}{\partial x} \cdot \frac{\partial x}{\partial \rho} + \frac{\partial f}{\partial y} \cdot \frac{\partial y}{\partial \rho} \\ &= \left(@dfdx@\right)\cdot\left(@dxdp@\right) + \left(@dfdy@\right)\cdot\left(@dydp@\right) \\ &=@factor(dfdx*dxdp+dfdy*dydp)@ \\ &= @subst([x=rho*cos(theta), y=rho*sin(theta)],factor(dfdx*dxdp+dfdy*dydp))@ \\ &= @ta2@ \end{align*}
\subsubsection{Question Note}
Find \( \frac{\partial f}{\partial \vartheta} \) and \( \frac{\partial f}{\partial \rho} \) of @f@ in polar coordinates.
\subsubsection{Feedback Variables}
\begin{lstlisting}
/* Feedback Variables */
notpolar1: dfdx*dxdt+dfdy*dydt;
notpolar2: dfdx*dxdp+dfdy*dydp;
\end{lstlisting}

\subsection{Differential Forms Monomials}
\subsubsection{Question Variables}
\begin{lstlisting}
powerdisp:true;

/* Exact function, f */
coef: rand_with_prohib(-3,3,[0]);
powx: rand_with_prohib(2,4,[0]);
powy: rand_with_prohib(2,4,[0]);
func: coef*x^(powx)*y^(powx);
fdx: diff(func,x);     /* Derive each term as diff from known function to ensure exactness */
fdy: diff(func,y);

/* Inexact function, g */
gdxcoef: rand_with_prohib(-9,9,[0]);
gdxpowx: rand_with_prohib(0,3,[0]);
gdxpowy: rand_with_prohib(0,3,[0]);
gdycoef: rand_with_prohib(-9,9,[0]);
gdypowx: rand_with_prohib(0,3,[0]);
gdypowy: rand_with_prohib(0,3,[0,gdxpowy-1]);     /* Prevents exactness */
gdx: gdxcoef*x^(gdxpowx)*y^(gdxpowy);
gdy: gdycoef*x^(gdypowx)*y^(gdypowy);

/* Randomisation of which Q appears first */
first: rand(2)+1;
second: rand_with_prohib(1,2,[first]);
dxpart: [fdx, gdx];
dypart: [fdy, gdy];

/* Teacher answers */
ta: [0,0,0,0,0,0];
ta[1]: diff(dxpart[first],y);
ta[2]: diff(dypart[first],x);
ta[3]: is(ta[1] = ta[2]);
ta[4]: diff(dxpart[second],y);
ta[5]: diff(dypart[second],x);
ta[6]: is(ta[4] = ta[5]);
\end{lstlisting}
\subsubsection{Question Text}
The following differential forms are in the form \(A \, \text{d}x + B \, \text{d}y\). Calculate the partial derivatives of \(A\) and \(B\) for each of the following differential forms, and hence decide whether they are exact or inexact differentials. Select true if it is an exact differential, and select false if it is an inexact differential.

(a) \(@dxpart[first]@ \, \text{d}x + @dypart[first]@ \, \text{d}y\) 

(i) \(\frac{\partial A}{\partial y} = \) [[input:ans1]][[validation:ans1]] 

(ii) \(\frac{\partial B}{\partial x} = \) [[input:ans2]][[validation:ans2]][[feedback:prt1]]

(iii) \(@dxpart[first]@ \, \text{d}x + @dypart[first]@ \, \text{d}y\) is an exact differential. [[input:ans3]][[validation:ans3]][[feedback:prt2]]

(b) \(@dxpart[second]@ \, \text{d}x + @dypart[second]@ \, \text{d}y\)

(i) \(\frac{\partial A}{\partial y} = \) [[input:ans4]][[validation:ans4]]

(ii) \(\frac{\partial B}{\partial x} = \) [[input:ans5]][[validation:ans5]][[feedback:prt3]]

(iii) \(@dxpart[second]@ \, \text{d}x + @dypart[second]@ \, \text{d}y\) is an exact differential. [[input:ans6]][[validation:ans6]][[feedback:prt4]]
\subsubsection{Hint}
Remember that a differential form of the form \(A \, \text{d}x + B \, \text{d}y\) is exact if and only if \(\frac{\partial A}{\partial y} = \frac{\partial B}{\partial x}\)
\subsubsection{Model Solution}
(a)(i) We differentiate \(A\) partially with respect to \(y\) by treating all other terms as constant \begin{align*} \frac{\partial A}{\partial y} &= \frac{\partial}{\partial y} \left( @dxpart[first]@ \right) \\ &= @dxpart[first]/(y^hipow(dxpart[first],y))@ \frac{\partial}{\partial y} \left( @y^hipow(dxpart[first],y)@ \right) \\ &= @ta[1]@ \end{align*}(ii) We differentiate \(B\) partially with respect to \(x\) by treating all other terms as constant \begin{align*} \frac{\partial B}{\partial x} &= \frac{\partial}{\partial x} \left( @dypart[first]@ \right) \\ &= @dypart[first]/(x^hipow(dypart[first],x))@ \frac{\partial}{\partial x} \left( @x^hipow(dypart[first],x)@ \right) \\ &= @ta[2]@ \end{align*}(iii) A differential form is exact if and only if \(\frac{\partial A}{\partial y} = \frac{\partial B}{\partial x}\). Hence, using the answers we just calculated, we see that it is @ta[3]@ that \(@dxpart[first]@ \, \text{d}x + @dypart[first]@ \, \text{d}y\) is an exact differential.

(b)(i) We differentiate \(A\) partially with respect to \(y\) by treating all other terms as constant \begin{align*} \frac{\partial A}{\partial y} &= \frac{\partial}{\partial y} \left( @dxpart[second]@ \right) \\ &= @dxpart[second]/(y^hipow(dxpart[second],y))@ \frac{\partial}{\partial y} \left( @y^hipow(dxpart[second],y)@ \right) \\ &= @ta[4]@ \end{align*}(ii) We differentiate \(B\) partially with respect to \(x\) by treating all other terms as constant \begin{align*} \frac{\partial B}{\partial x} &= \frac{\partial}{\partial x} \left( @dypart[second]@ \right) \\ &= @dypart[second]/(x^hipow(dypart[second],x))@ \frac{\partial}{\partial x} \left( @x^hipow(dypart[second],x)@ \right) \\ &= @ta[5]@ \end{align*}(iii) A differential form is exact if and only if \(\frac{\partial A}{\partial y} = \frac{\partial B}{\partial x}\). Hence, using the answers we just calculated, we see that it is @ta[6]@ that \(@dxpart[second]@ \, \text{d}x + @dypart[second]@ \, \text{d}y\) is an exact differential.
\subsubsection{Question Note}
Calculate the partial derivatives and decide whether they are exact or inexact differentials for \(@dxpart[first]@ \, \text{d}x + @dypart[first]@ \, \text{d}y\)  and \(@dxpart[second]@ \, \text{d}x + @dypart[second]@ \, \text{d}y\).

\subsection{Differential Forms Non-Monomials}
\subsubsection{Question Variables}
\begin{lstlisting}
powerdisp:true;
funcslist: [sin,cos,exp];

/* Exact function, f */
fnumb: rand(3)+1;
coef: rand_with_prohib(-3,3,[0]);
powx: rand(2)+1;
powy: if (powx=1) then rand(2)+1 else 1;
func: coef*funcslist[fnumb](x^(powx)*y^(powy));
fdx: diff(func,x);     /* Derive each term as diff from known function to ensure exactness */
fdy: diff(func,y);

/* Inexact function, g */
gnumb: rand(3)+1;
gdxcoef: rand_with_prohib(-3,3,[0]);
gdxpowx: rand_with_prohib(0,2,[0]);
gdxpowy: if (gdxpowx=1) then rand(2)+1 else 1;
gdycoef: rand_with_prohib(-3,3,[0]);
gdypowx: rand_with_prohib(0,2,[0]);
gdypowy: if (gdypowx=1) then rand(2)+1 else 1;
gdx: gdxcoef*gdxpowy*x^(gdxpowx )*y^(gdxpowy -1)*funcslist[gnumb](x^(gdxpowx)*y^(gdxpowy));
gdy: gdycoef*gdypowx *x^(gdypowx -1)*y^(gdypowy)*funcslist[gnumb](x^(gdypowx)*y^(gdypowy));     /* these can be proven to never create an exact diff form (yet hard to tell at first glance) */

/* Randomisation of which Q appears first */
first: rand(2)+1;
second: rand_with_prohib(1,2,[first]);
dxpart: [fdx, gdx];
dypart: [fdy, gdy];

/* Teacher answers */
ta: [0,0,0,0,0,0];
ta[1]: factor(diff(dxpart[first],y));
ta[2]:  factor(diff(dypart[first],x));
ta[3]: is(ta[1] = ta[2]);
ta[4]:  factor(diff(dxpart[second],y));
ta[5]:  factor(diff(dypart[second],x));
ta[6]: is(ta[4] = ta[5]);
\end{lstlisting}
\subsubsection{Question Text}
The following differential forms are in the form $A \, \text{d}x + B \, \text{d}y$. Calculate the partial derivatives of $A$ and $B$ for each of the following differential forms, and hence decide whether they are exact or inexact differentials. Select true if it is an exact differential, and select false if it is an inexact differential.

(a) $@dxpart[first]@ \, \text{d}x + @dypart[first]@ \, \text{d}y$ 

(i) $\frac{\partial A}{\partial y} = $ [[input:ans1]][[validation:ans1]] 

(ii) $\frac{\partial B}{\partial x} = $ [[input:ans2]][[validation:ans2]][[feedback:prt1]]

(iii) $@dxpart[first]@ \, \text{d}x + @dypart[first]@ \, \text{d}y$ is an exact differential. [[input:ans3]][[validation:ans3]][[feedback:prt2]]

(b) $@dxpart[second]@ \, \text{d}x + @dypart[second]@ \, \text{d}y$

(i) $\frac{\partial A}{\partial y} = $ [[input:ans4]][[validation:ans4]]

(ii) $\frac{\partial B}{\partial x} = $ [[input:ans5]][[validation:ans5]][[feedback:prt3]]

(iii) $@dxpart[second]@ \, \text{d}x + @dypart[second]@ \, \text{d}y$ is an exact differential. [[input:ans6]][[validation:ans6]][[feedback:prt4]]
\subsubsection{Hint}
Remember that a differential form of the form $A \, \text{d}x + B \, \text{d}y$ is exact if and only if $\frac{\partial A}{\partial y} = \frac{\partial B}{\partial x}$
\subsubsection{Model Solution}
(a)(i) We differentiate $A$ partially with respect to $y$ by treating all other terms as constant \begin{align*} \frac{\partial A}{\partial y} &= \frac{\partial}{\partial y} \left( @dxpart[first]@ \right) \\ &= @diff(dxpart[first],y)@  \\ &= @ta[1]@ \end{align*}(ii) We differentiate $B$ partially with respect to $x$ by treating all other terms as constant \begin{align*} \frac{\partial B}{\partial x} &= \frac{\partial}{\partial x} \left( @dypart[first]@ \right) \\ &= @diff(dypart[first],x)@  \\ &= @ta[2]@ \end{align*}(iii) A differential form is exact if and only if $\frac{\partial A}{\partial y} = \frac{\partial B}{\partial x}$. Hence, using the answers we just calculated, we see that it is @ta[3]@ that $@dxpart[first]@ \, \text{d}x + @dypart[first]@ \, \text{d}y$ is an exact differential.

(b)(i) We differentiate $A$ partially with respect to $y$ by treating all other terms as constant \begin{align*} \frac{\partial A}{\partial y} &= \frac{\partial}{\partial y} \left( @dxpart[second]@ \right) \\ &= @diff(dxpart[second],y)@  \\ &= @ta[4]@ \end{align*}(ii) We differentiate $B$ partially with respect to $x$ by treating all other terms as constant \begin{align*} \frac{\partial B}{\partial x} &= \frac{\partial}{\partial x} \left( @dypart[second]@ \right) \\ &= @diff(dypart[second],x)@   \\ &= @ta[5]@ \end{align*}(iii) A differential form is exact if and only if $\frac{\partial A}{\partial y} = \frac{\partial B}{\partial x}$. Hence, using the answers we just calculated, we see that it is @ta[6]@ that $@dxpart[second]@ \, \text{d}x + @dypart[second]@ \, \text{d}y$ is an exact differential.
\subsubsection{Question Note}
Calculate the partial derivatives and decide whether they are exact or inexact differentials for $@dxpart[first]@ \, \text{d}x + @dypart[first]@ \, \text{d}y$  and $@dxpart[second]@ \, \text{d}x + @dypart[second]@ \, \text{d}y$.

\subsection{Differential Forms Non-Monomials 2}
\subsubsection{Question Variables}
\begin{lstlisting}
powerdisp:true;
funcslist: [sin,cos,exp];

/* Exact function, f */
fnumbx: rand(3)+1;
fnumby: rand(3)+1;
coefx: rand_with_prohib(-3,3,[0]);
coefy: rand_with_prohib(-3,3,[0]);
func: funcslist[fnumbx](coefx*x)*funcslist[fnumby](coefy*y);
fdx: diff(func,x);     /* Derive each term as diff from known function to ensure exactness */
fdy: diff(func,y);

/* Inexact function, g */
gxnumbdx: rand(2)+1;
gynumbdx: rand(2)+1;
gxnumbdy: rand(2)+1;
gynumbdy: rand(2)+1;
gcoefx: rand_with_prohib(-3,3,[0]);
gcoefy: rand_with_prohib(-3,3,[0]);
gdxcoef: gcoefy;
gdycoef: gcoefx;
gdx: gdxcoef*funcslist[gxnumbdx](gcoefx*x)*funcslist[gynumbdx](gcoefy*y);
gdy: gdycoef*funcslist[gxnumbdy](gcoefx*x)*funcslist[gynumbdy](gcoefy*y);     /* these can be proven to never create an exact diff form (yet hard to tell at first glance) */

/* Randomisation of which Q appears first */
first: rand(2)+1;
second: rand_with_prohib(1,2,[first]);
dxpart: [fdx, gdx];
dypart: [fdy, gdy];

/* Teacher answers */
ta: [0,0,0,0,0,0];
ta[1]: diff(dxpart[first],y);
ta[2]:  diff(dypart[first],x);
ta[3]: is(ta[1] = ta[2]);
ta[4]: diff(dxpart[second],y);
ta[5]:  diff(dypart[second],x);
ta[6]: is(ta[4] = ta[5]);
\end{lstlisting}
\subsubsection{Question Text}
The following differential forms are in the form \(A \, \text{d}x + B \, \text{d}y\). Calculate the partial derivatives of \(A\) and \(B\) for each of the following differential forms, and hence decide whether they are exact or inexact differentials. Select true if it is an exact differential, and select false if it is an inexact differential.

(a) \(@dxpart[first]@ \, \text{d}x + @dypart[first]@ \, \text{d}y\) 

(i) \(\frac{\partial A}{\partial y} = \) [[input:ans1]][[validation:ans1]] 

(ii) \(\frac{\partial B}{\partial x} = \) [[input:ans2]][[validation:ans2]][[feedback:prt1]]

(iii) \(@dxpart[first]@ \, \text{d}x + @dypart[first]@ \, \text{d}y\) is an exact differential. [[input:ans3]][[validation:ans3]][[feedback:prt2]]

(b) \(@dxpart[second]@ \, \text{d}x + @dypart[second]@ \, \text{d}y\)

(i) \(\frac{\partial A}{\partial y} = \) [[input:ans4]][[validation:ans4]]

(ii) \(\frac{\partial B}{\partial x} = \) [[input:ans5]][[validation:ans5]][[feedback:prt3]]

(iii) \(@dxpart[second]@ \, \text{d}x + @dypart[second]@ \, \text{d}y\) is an exact differential. [[input:ans6]][[validation:ans6]][[feedback:prt4]]
\subsubsection{Hint}
Remember that a differential form of the form \(A \, \text{d}x + B \, \text{d}y\) is exact if and only if \(\frac{\partial A}{\partial y} = \frac{\partial B}{\partial x}\)
\subsubsection{Model Solution}
(a)(i) We differentiate \(A\) partially with respect to \(y\) by treating all other terms as constant \begin{align*} \frac{\partial A}{\partial y} &= \frac{\partial}{\partial y} \left( @dxpart[first]@ \right) \\ &= @ta[1]@ \end{align*}(ii) We differentiate \(B\) partially with respect to \(x\) by treating all other terms as constant \begin{align*} \frac{\partial B}{\partial x} &= \frac{\partial}{\partial x} \left( @dypart[first]@ \right) \\ &= @ta[2]@ \end{align*}(iii) A differential form is exact if and only if \(\frac{\partial A}{\partial y} = \frac{\partial B}{\partial x}\). Hence, using the answers we just calculated, we see that it is @ta[3]@ that \(@dxpart[first]@ \, \text{d}x + @dypart[first]@ \, \text{d}y\) is an exact differential.

(b)(i) We differentiate \(A\) partially with respect to \(y\) by treating all other terms as constant \begin{align*} \frac{\partial A}{\partial y} &= \frac{\partial}{\partial y} \left( @dxpart[second]@ \right) \\ &= @ta[4]@ \end{align*}(ii) We differentiate \(B\) partially with respect to \(x\) by treating all other terms as constant \begin{align*} \frac{\partial B}{\partial x} &= \frac{\partial}{\partial x} \left( @dypart[second]@ \right) \\ &= @ta[5]@ \end{align*}(iii) A differential form is exact if and only if \(\frac{\partial A}{\partial y} = \frac{\partial B}{\partial x}\). Hence, using the answers we just calculated, we see that it is @ta[6]@ that \(@dxpart[second]@ \, \text{d}x + @dypart[second]@ \, \text{d}y\) is an exact differential.
\subsubsection{Question Note}
Calculate the partial derivatives and decide whether they are exact or inexact differentials for \(@dxpart[first]@ \, \text{d}x + @dypart[first]@ \, \text{d}y\)  and \(@dxpart[second]@ \, \text{d}x + @dypart[second]@ \, \text{d}y\).

\subsection{Directional Derivatives}
\subsubsection{Question Variables}
\begin{lstlisting}
powerdisp:true;

/* Functions */
Grad(f) := diff(f,x)*i + diff(f,y)*j + diff(f,z)*k;
dot(f,g) := coeff(f,i)*coeff(g,i)+coeff(f,j)*coeff(g,j)+coeff(f,k)*coeff(g,k);
mag(v) := sqrt(coeff(v,i)^2 + coeff(v,j)^2 + coeff(v,k)^2);
dirderiv(f,u) := dot(f,u)/mag(u);

/* Defining function phi */
FuncList:[sin,cos,exp,log];
Func:[rand(4)+1,0,0];
Func[2]:rand_with_prohib(1,4,[Func[1]]);
Func[3]:rand_with_prohib(1,4,[Func[1],Func[2]]);
coefx: rand_with_prohib(0,3,[0]);
coefy: rand_with_prohib(0,3,[0]);
coefz: rand_with_prohib(0,3,[0]);
phi: FuncList[Func[1]](x*coefx)*rand_with_prohib(-3,3,[0]) + FuncList[Func[2]](y*coefy)*rand_with_prohib(-3,3,[0]) + FuncList[Func[3]](coefz*z)*rand_with_prohib(-3,3,[0]);

/* Co-ordinates */
NiceCoords:[[0,pi/3,pi/2,2*pi/3,pi,-pi/3,-pi/2,-2*pi/3,-pi],[0,pi/6,5*pi/6,pi,-pi/6,-5*pi/6,-pi], [0,log(2),log(3)], [1,2,3,4,5]];
Coords:[rand(NiceCoords[Func[1]])/coefx, rand(NiceCoords[Func[2]])/coefy, rand(NiceCoords[Func[3]])/coefz];      /* Dividing through by coef to ensure nice answers are given. Does not allow zero gradient by choice of co-ordinates */

/* Directions */
Direction1: rand([i,j,k]);
Direction2: rand([i+j+k,i+j-k,i-j+k,i-j-k,-i+j+k,-i+j-k,-i-j+k,-i-j-k]);

/* Teacher Answers */
ta1: Grad(phi);
ta2: subst( [x=Coords[1], y=Coords[2], z=Coords[3]], dirderiv(ta1,Direction1));
ta3: subst( [x=Coords[1], y=Coords[2], z=Coords[3]], dirderiv(ta1,Direction2));
\end{lstlisting}
\subsubsection{Question Text}
Consider the scalar function \[\phi = @phi@\](a) Using the suggested form for your solution provided, calculate \(\nabla \phi\). Note that \(e^{x}\) can be typed as exp(x) in the answer box.

\(\nabla \phi = \) [[input:ans1]][[validation:ans1]][[feedback:prt1]]

(b) Calculate the directional derivative of $\phi$ along the direction \(\mathbf u\), denoted by \(\nabla_{\textbf{u}} \phi\), at the point \( (x,y,z) = \left(@Coords[1]@, @Coords[2]@, @Coords[3]@\right)\) for the following choices of \(\textbf{u}\). You may answer in either symbols or decimal form, and do not need to give your answer in its simplest form. Note that \(\sqrt{n}\) can be typed assqrt(n) in the answer box. If you choose to answer in decimal form, give your answer to at least two decimal places.

(i) \(\mathbf u = \mathbf{@Direction1@}\)

\(\nabla_{\textbf{u}} \phi =\) [[input:ans2]][[validation:ans2]][[feedback:prt2]]

(ii) \(\mathbf u = \mathbf{@Direction2@}\)

\(\nabla_{\textbf{u}} \phi =\) [[input:ans3]][[validation:ans3]][[feedback:prt3]]
\subsubsection{Hint}
Remember the directional derivative of \(\phi\) at \((x,y,z)\) in direction \(\mathbf{u}\) is given by \begin{align*} \nabla_{\textbf{u}}\phi(x,y,z) = \nabla\phi(x,y,z) \cdot \frac{\textbf{u}}{|\textbf{u}|} \end{align*}
\subsubsection{Model Solution}
(a) We calculate \begin{align*} \nabla\phi &= \frac{\partial \phi}{\partial x}\mathbf i + \frac{\partial \phi}{\partial y}\mathbf j + \frac{\partial \phi}{\partial z}\mathbf k\\ &= @diff(phi,x)@ \, \mathbf i + @diff(phi,y)@ \, \mathbf j + @diff(phi,z)@ \, \mathbf k \end{align*}

(b) Recall the formula \begin{align*} \nabla_{\textbf{u}}\phi(x,y,z) = \nabla\phi(x,y,z) \cdot \frac{\textbf{u}}{|\textbf{u}|} \end{align*} So, evaluating \(\phi\) at \( (x,y,z) = \left(@Coords[1]@, @Coords[2]@, @Coords[3]@\right)\) gives us \begin{align*} \nabla \phi \left(@Coords[1]@, @Coords[2]@, @Coords[3]@\right) = @subst([x=Coords[1],y=Coords[2],z=Coords[3]], diff(phi,x))@ \, \mathbf i + @subst([x=Coords[1],y=Coords[2],z=Coords[3]], diff(phi,y))@ \, \mathbf j + @subst([x=Coords[1],y=Coords[2],z=Coords[3]], diff(phi,z))@ \, \mathbf k \end{align*} Hence, we can proceed to calculate the directional derivatives by taking the dot product of this vector with the normalised vector of \(\textbf{u}\).

(i) Since \(\textbf{u}\) is already a unit vector, we do not need to concern ourselves with any normalisation, and can proceed using $\textbf{u}$, as it is equal to $\frac{\textbf{u}}{|\textbf{u}|}$. \begin{align*} \nabla_{\textbf{u}}\phi\left(@Coords[1]@, @Coords[2]@, @Coords[3]@\right) &= \left(@subst([x=Coords[1],y=Coords[2],z=Coords[3]], diff(phi,x))@ \, \mathbf i + @subst([x=Coords[1],y=Coords[2],z=Coords[3]], diff(phi,y))@ \, \mathbf j + @subst([x=Coords[1],y=Coords[2],z=Coords[3]], diff(phi,z))@ \, \mathbf k \right) \cdot \left(@coeff(Direction1,i)@ \, \textbf{i} + @coeff(Direction1,j)@ \, \textbf{j} + @coeff(Direction1,k)@ \, \textbf{k} \right) \\ &= @subst([x=Coords[1],y=Coords[2],z=Coords[3]], diff(phi,x))@\cdot@coeff(Direction1,i)@ + @subst([x=Coords[1],y=Coords[2],z=Coords[3]], diff(phi,y))@\cdot@coeff(Direction1,j)@ + @subst([x=Coords[1],y=Coords[2],z=Coords[3]], diff(phi,z))@ \\ &= @ta2@ \end{align*}

(ii) We first calculate the magnitude of \(\textbf{u}\) \begin{align*} \mathbf u = \mathbf{@Direction2@} \quad \Rightarrow \quad |\textbf{u}|=\sqrt{(@coeff(Direction2,i)@)^2 + (@coeff(Direction2,j)@)^2 + (@coeff(Direction2,k)@)^2}=@mag(Direction2)@ \end{align*} Then, we compute the dot product using the normalised vector \begin{align*} \nabla_{\textbf{u}}\phi\left(@Coords[1]@, @Coords[2]@, @Coords[3]@\right) &= \left(@subst([x=Coords[1],y=Coords[2],z=Coords[3]], diff(phi,x))@ \, \mathbf i + @subst([x=Coords[1],y=Coords[2],z=Coords[3]], diff(phi,y))@ \, \mathbf j + @subst([x=Coords[1],y=Coords[2],z=Coords[3]], diff(phi,z))@ \, \mathbf k \right) \cdot \frac{1}{@mag(Direction2)@} \left(@coeff(Direction2,i)@ \, \textbf{i} + @coeff(Direction2,j)@ \, \textbf{j} + @coeff(Direction2,k)@ \, \textbf{k} \right) \\ &= \frac{1}{@mag(Direction2)@} \left( @subst([x=Coords[1],y=Coords[2],z=Coords[3]], diff(phi,x))@\cdot@coeff(Direction2,i)@ + @subst([x=Coords[1],y=Coords[2],z=Coords[3]], diff(phi,y))@\cdot@coeff(Direction2,j)@ + @subst([x=Coords[1],y=Coords[2],z=Coords[3]], diff(phi,z))@\cdot@coeff(Direction2,k)@ \right) \\ &= @ta3@
\end{align*}
\subsubsection{Question Note}
Calculate the gradient of \(\phi = @phi@\) and the directional derivatives at \([x,y,z]=@Coords@\) in directions @Direction1@ and @Direction2@.
\subsubsection{Feedback Variables}
\begin{lstlisting}
/* Feedback Variables */
ecf2: subst( [x=Coords[1], y=Coords[2], z=Coords[3]], dirderiv(ans1,Direction1));
notnormal3: subst(Coords[1],x,subst(Coords[2],y,subst(Coords[3],z,dot(ta1,Direction2))));
ecf3: subst( [x=Coords[1], y=Coords[2], z=Coords[3]], dirderiv(ta1,Direction2));
\end{lstlisting}

\subsection{Partial Derivatives}
\subsubsection{Question Variables}
\begin{lstlisting}
powerdisp:true;

/* Creating the function */
func: rand([cos,sin,exp]);
coef: rand_with_prohib(-5,5,[0]);
pows: [0,1,1,2];
xpowgen1: rand_with_prohib(1,4,[0]);
xpowgen2: rand_with_prohib(1,4,[xpowgen1]);
ypowgen1: rand_with_prohib(1,4,[xpowgen1,xpowgen2]);
ypowgen2: rand_with_prohib(1,4,[xpowgen1,xpowgen2,ypowgen1]);
xpow1: pows[xpowgen1];
xpow2: pows[xpowgen2];
ypow1: pows[ypowgen1];
ypow2: pows[ypowgen2];
f: coef*x^(xpow1)*y^(ypow1)*func((x^(xpow2)*y^(ypow2)));

/* Teacher Answers */
ta: [0,0,0,0,0];
ta[1]: factor(trigsimp(diff(f,x)));
ta[2]: factor(trigsimp(diff(f,y)));
ta[3]: factor(trigsimp(diff(f,x,2)));
ta[4]: factor(trigsimp(diff(f,y,2)));
ta[5]: factor(trigsimp(diff(diff(f,x),y)));
\end{lstlisting}
\subsubsection{Question Text}
For the function \[  f(x,y)=@coef*x^(xpow1)*y^(ypow1)@ \, @func((x^(xpow2)*y^(ypow2)))@ \] obtain the following partial derivatives:

(a) \( \frac{\partial f}{\partial x} = \) [[input:ans1]][[validation:ans1]][[feedback:prt1]]

(b) \( \frac{\partial f}{\partial y} = \) [[input:ans2]][[validation:ans2]][[feedback:prt2]]

(c) \( \frac{\partial^2 f}{\partial x^2} = \) [[input:ans3]][[validation:ans3]][[feedback:prt3]]

(d) \( \frac{\partial^2 f}{\partial y^2} = \) [[input:ans4]][[validation:ans4]][[feedback:prt4]]

(e) \( \frac{\partial^2 f}{\partial x \partial y} = \frac{\partial^2 f}{\partial y \partial x} = \) [[input:ans5]][[validation:ans5]][[feedback:prt5]]
\subsubsection{Hint}
Remember that partially differentiating with respect to a variable is the same as treating all other variables as constant, and performing regular differentiation with respect to the variable.
\subsubsection{Model Solution}
Partially differentiating with respect to a variable treats all other variables as constant.

(a) Partially differentiating with respect to \(x\), we apply the product rule to get \begin{align*} \frac{\partial f}{\partial x} &= \frac{\partial}{\partial x} \left( @coef*x^(xpow1)*y^(ypow1)@ \, @func((x^(xpow2)*y^(ypow2)))@ \right) \\ &= @diff(f,x)@ \\ &= @ta[1]@ \end{align*}(b) Partially differentiating with respect to \(y\), we apply the product rule to get \begin{align*} \frac{\partial f}{\partial y} &= \frac{\partial}{\partial y} \left( @coef*x^(xpow1)*y^(ypow1)@ \, @func((x^(xpow2)*y^(ypow2)))@ \right) \\ &= @diff(f,y)@ \\ &= @ta[2]@ \end{align*}(c) To partially differentiate with respect to \(x\) twice, we partially differentiate \(\frac{\partial f}{\partial x}\) with respect to \(x\) \begin{align*} \frac{\partial^2 f}{\partial x^2} &= \frac{\partial}{\partial x} \left( \frac{\partial f}{\partial x} \right) \\ &= \frac{\partial}{\partial x} \left( @ta[1]@ \right) \\ &= @diff(f,x,2)@ \\ &= @ta[3]@ \end{align*}(d) To partially differentiate with respect to \(y\) twice, we partially differentiate \(\frac{\partial f}{\partial y}\) with respect to \(y\) \begin{align*} \frac{\partial^2 f}{\partial y^2} &= \frac{\partial}{\partial y} \left( \frac{\partial f}{\partial y} \right) \\ &= \frac{\partial}{\partial y} \left( @ta[2]@ \right) \\ &= @diff(f,y,2)@ \\ &= @ta[4]@ \end{align*} (e) To find \(\frac{\partial^2 f}{\partial x \partial y}\), we partially differentiate \(\frac{\partial f}{\partial y}\) with respect to \(x\) \begin{align*} \frac{\partial^2 f}{\partial x \partial y} &= \frac{\partial}{\partial x} \left( \frac{\partial f}{\partial y} \right) \\ &= \frac{\partial}{\partial x} \left( @ta[2]@ \right) \\ &= @diff(diff(f,y),x)@ \\ &= @ta[5]@ \end{align*} You could also find \(\frac{\partial^2 f}{\partial y \partial x}\) by partially differentiating \(\frac{\partial f}{\partial x}\) with respect to \(y\) to get the same solution.
\subsubsection{Question Note}
Find all the first and second order partial derivatives of \(@f@\).
\subsubsection{Feedback Variables}
\begin{lstlisting}
/* Feedback Variables */
ecf3: diff(ans1,x);
ecf4: diff(ans2,y);
ecf51: diff(ans1,y);
ecf52: diff(ans2,x);
\end{lstlisting}

\section{Test 2}

\subsection{Angular Velocity}
\subsubsection{Question Variables}
\begin{lstlisting}
powerdisp:true;

/* Functions */
Curl(f) := (diff(coeff(f,k),y)-diff(coeff(f,j),z))*i + (diff(coeff(f,i),z)-diff(coeff(f,k),x))*j + (diff(coeff(f,j),x)-diff(coeff(f,i),y))*k;
Cross(a,b) := (coeff(a,j)*coeff(b,k)-coeff(a,k)*coeff(b,j))*i  + (coeff(a,k)*coeff(b,i)-coeff(a,i)*coeff(b,k))*j + (coeff(a,i)*coeff(b,j)-coeff(a,j)*coeff(b,i))*k;

/* Variables */
omega: (rand_with_prohib(-5,5,[0]))*i+(rand(11)-5)*j+(rand(11)-5)*k;
position: x*i+y*j+z*k;
vel: Cross(omega,position);

/* Teacher Answer */
ta1: Curl(vel);
\end{lstlisting}
\subsubsection{Question Text}
A solid object rotates about the \(z\)-axis with angular velocity \( \boldsymbol{\omega} = @coeff(omega,i)@ \, \mathbf{i} + @coeff(omega,j)@ \, \mathbf{j} + @coeff(omega,k)@ \, \mathbf{k} \).

Calculate the curl of the velocity vector field, \(\mathbf{v}\).

Hint: Remember that \(\mathbf{v} =\boldsymbol{\omega} \times \mathbf{r}\), and you can take \(\mathbf{r} = x \, \mathbf{i} + y \, \mathbf{j} + z \, \mathbf{k}\). Use the suggested form for your answer given.

\(\nabla \times \mathbf{v} =\) [[input:ans1]][[validation:ans1]][[feedback:prt1]]
\subsubsection{Hint}
Try taking the cross product of \(\boldsymbol{\omega}\) and \(\mathbf{r}\) using the formula for \(\mathbf{r}\) provided. This will give you the vector field \(\mathbf{v}\), which you can then find the curl of.
\subsubsection{Model Solution}
Firstly, we find the vector field \(\mathbf{v}\) by taking the cross product of \(\boldsymbol{\omega}\) and \(\mathbf{r}\) \begin{align*} \mathbf{v} &=\boldsymbol{\omega} \times \mathbf{r} \\ &= \begin{vmatrix} \mathbf{i} & \mathbf{j} & \mathbf{k} \\ @coeff(omega,i)@ & @coeff(omega,j)@ & @coeff(omega,k)@ \\ x & y & z \end{vmatrix} \\ &= (@coeff(omega,j)@ \cdot z -@coeff(omega,k)@ \cdot y ) \, \mathbf{i} - (@coeff(omega,i)@ \cdot z -@coeff(omega,k)@ \cdot x ) \, \mathbf{j} + (@coeff(omega,i)@ \cdot y -@coeff(omega,j)@ \cdot x ) \, \mathbf{k} \\ &= \left(@coeff(vel,i)@\right) \, \mathbf{i} + \left(@coeff(vel,j)@\right) \, \mathbf{j} + \left(@coeff(vel,k)@\right) \, \mathbf{k} \end{align*} All the remains is to take the curl of this vector field \begin{align*} \nabla \times \mathbf{v} &= \begin{vmatrix} \mathbf{i} & \mathbf{j} & \mathbf{k} \\ \frac{\partial}{\partial x} & \frac{\partial}{\partial y} & \frac{\partial}{\partial z} \\ @coeff(vel,i)@ & @coeff(vel,j)@ & @coeff(vel,k)@ \end{vmatrix} \\ &= \left(@diff(coeff(vel,k),y)@-@diff(coeff(vel,j),z)@\right) \, \textbf{i} - \left(@diff(coeff(vel,k),x)@-@diff(coeff(vel,i),z)@\right) \, \textbf{j} + \left(@diff(coeff(vel,j),x)@-@diff(coeff(vel,i),y)@\right) \, \textbf{k} \\ &= @coeff(ta1,i)@ \, \textbf{i} + @coeff(ta1,j)@ \, \textbf{j} + @coeff(ta1,k)@ \, \textbf{k} \end{align*}
\subsubsection{Question Note}
Calculate the curl of velocity vector field \(\mathbf{v}\) for an object rotating around the \(z\)-axis with angular velocity \(\boldsymbol{\omega} = @coef@\mathbf{k}\).

\subsection{Curl of Gradient Field}
\subsubsection{Question Variables}
\begin{lstlisting}
powerdisp:true;

ta1: 0;     /* Simple maths shows the answer is always 0 */
\end{lstlisting}
\subsubsection{Question Text}
Calculate $\nabla \times \left( \nabla \phi \right)$, where $\phi$ is a scalar. Use the suggested form for your solution which has been given.

$\nabla \times \left( \nabla \phi \right) = $ [[input:ans1]][[validation:ans1]][[feedback:prt1]]
\subsubsection{Hint}
Recall the formula \begin{align*} \nabla\phi &= \frac{\partial \phi}{\partial x}\mathbf i + \frac{\partial \phi}{\partial y}\mathbf j + \frac{\partial \phi}{\partial z}\mathbf k \end{align*} Using this general form, you should calculate the curl of it to get the solution.
\subsubsection{Model Solution}
We have \begin{align*} \nabla\phi &= \frac{\partial \phi}{\partial x}\mathbf i + \frac{\partial \phi}{\partial y}\mathbf j + \frac{\partial \phi}{\partial z}\mathbf k \end{align*} Hence, calculating the curl of this, we get \begin{align*} \nabla \times \left( \nabla \phi \right) &= \begin{vmatrix} \mathbf{i} & \mathbf{j} & \mathbf{k} \\ \frac{\partial }{\partial x} & \frac{\partial}{\partial y} & \frac{\partial}{\partial z} \\ \frac{\partial \phi}{\partial x} & \frac{\partial \phi}{\partial y} & \frac{\partial \phi}{\partial z} \end{vmatrix} \\ &= \left( \frac{\partial}{\partial y}\frac{\partial \phi}{\partial z} - \frac{\partial}{\partial z}\frac{\partial \phi}{\partial y} \right) \textbf{i} - \left( \frac{\partial}{\partial x}\frac{\partial \phi}{\partial z} - \frac{\partial}{\partial z}\frac{\partial \phi}{\partial x} \right) \textbf{j} + \left( \frac{\partial}{\partial x}\frac{\partial \phi}{\partial y} - \frac{\partial}{\partial y}\frac{\partial \phi}{\partial x} \right) \textbf{k} \end{align*} But remember that \[ \frac{\partial}{\partial x}\frac{\partial \phi}{\partial y} = \frac{\partial^2 \phi}{\partial x \partial y} = \frac{\partial^2 \phi}{\partial y \partial x} = \frac{\partial}{\partial y}\frac{\partial \phi}{\partial x} \] This is also holds when replacing $x$ and $y$ with any of the other variables $x$, $y$ and $z$. So, \begin{align*} \nabla \times \left( \nabla \phi \right) &= \left( \frac{\partial}{\partial y}\frac{\partial \phi}{\partial z} - \frac{\partial}{\partial z}\frac{\partial \phi}{\partial y} \right) \textbf{i} - \left( \frac{\partial}{\partial x}\frac{\partial \phi}{\partial z} - \frac{\partial}{\partial z}\frac{\partial \phi}{\partial x} \right) \textbf{j} + \left( \frac{\partial}{\partial x}\frac{\partial \phi}{\partial y} - \frac{\partial}{\partial y}\frac{\partial \phi}{\partial x} \right) \textbf{k} \\ &= \underbrace{\left( \frac{\partial^2 \phi}{\partial y \partial z}- \frac{\partial^2 \phi}{\partial z \partial y} \right)}_{=0} \textbf{i} - \underbrace{\left( \frac{\partial^2 \phi}{\partial x \partial z} - \frac{\partial^2 \phi}{\partial z \partial x} \right)}_{=0} \textbf{j} + \underbrace{\left(\frac{\partial^2 \phi}{\partial x \partial y} - \frac{\partial^2 \phi}{\partial y \partial x} \right)}_{=0} \textbf{k} \end{align*} Hence, $\nabla \times \left( \nabla \phi \right) = 0$.
\subsubsection{Question Note}
Calculate the curl of the gradient field of a scalar.

\subsection{Differential Operators Example}
\subsubsection{Question Variables}
\begin{lstlisting}
powerdisp:true;

/* Functions */
Curl(f) := (diff(coeff(f,k),y)-diff(coeff(f,j),z))*i + (diff(coeff(f,i),z)-diff(coeff(f,k),x))*j + (diff(coeff(f,j),x)-diff(coeff(f,i),y))*k;
Div(f) := diff(coeff(f,i),x) + diff(coeff(f,j),y) + diff(coeff(f,k),z);
Grad(f) := diff(f,x)*i + diff(f,y)*j + diff(f,z)*k;
Grad2(f) := diff(f,x,2) + diff(f,y,2) + diff(f,z,2);

/* Defining vector field a. It has two directions using cos/sin/exp and one using polynomials. It is designed so the divergence is never zero using special(i/j/k) to multiply the polynomial term by a single variable x/y/z depending on if it is in i/j/k */
avectfun: [0,0,0];
fun1: rand([cos,sin,exp]);
fun2: rand([cos,sin,exp]);
avectfun[1]: fun1(rand([x,y,z]));     /* choosing cos/sin/exp */
avectfun[2]: fun2(rand([x,y,z]));
avectfun[3]: (rand([x,y,z]))^(rand(2)+1)+(rand([x,y,z]))^(rand(2)+1);     /* choosing polynomial */
avectinumb: rand(3)+1;     /* number to pick which function in i */
avectjnumb: rand_with_prohib(1,3,[avectinumb]);
avectknumb: rand_with_prohib(1,3,[avectinumb,avectjnumb]);
speciali: if (avectinumb = 3) then x else 1;     /* if polynomial in i, multiply by extra term */
specialj: if (avectjnumb = 3) then y else 1;
specialk: if (avectknumb = 3) then z else 1;
avect: speciali*avectfun[avectinumb]*i + specialj*avectfun[avectjnumb]*j + specialk*avectfun[avectknumb]*k;     /* defines a */

/* Teacher Answers */
ta1: trigsimp(Div(avect));
ta2: trigsimp(Grad(Div(avect)));
ta3: trigsimp(Grad2(avect));
ta4: trigsimp(Curl(avect));
ta5: is(cabs(coeff(Curl(avect),i))=0) and is(cabs(coeff(Curl(avect),j))=0) and is(cabs(coeff(Curl(avect),k))=0);
\end{lstlisting}
\subsubsection{Question Text}
Consider the vector field, \(\mathbf{a}\), given by \[ \mathbf{a} = \left(@coeff(avect,i)@\right) \, \mathbf{i}+\left(@coeff(avect,j)@\right) \, \mathbf{j}+ \left(@coeff(avect,k)@\right) \, \mathbf{k}\]Calculate the following, using the suggested form for your answer where it has been provided.

(a) \(\nabla \cdot \mathbf{a} = \) [[input:ans1]][[validation:ans1]][[feedback:prt1]]

(b) \(\nabla(\nabla \cdot \mathbf{a}) = \) [[input:ans2]][[validation:ans2]][[feedback:prt2]]

(c) \(\nabla^2 \mathbf{a} = \) [[input:ans3]][[validation:ans3]][[feedback:prt3]]

(d)  \(\nabla \times \mathbf{a} = \) [[input:ans4]][[validation:ans4]][[feedback:prt4]]

(e) There exists a function, \(\phi\), such that \(\mathbf{a} = \nabla \phi\). [[input:ans5]][[validation:ans5]][[feedback:prt5]]
\subsubsection{Hint}
All the questions should allow you to apply the general form to our given vector field. If you have not seen the notation $\nabla^2 \mathbf{a}$ before, then note \begin{align*} \nabla^2 \mathbf{a} &= \left(\frac{\partial^2}{\partial x^2} + \frac{\partial^2}{\partial y^2} + \frac{\partial^2}{\partial z^2}\right)\textbf{a} \\ &= \left(\frac{\partial^2}{\partial x^2} + \frac{\partial^2}{\partial y^2} + \frac{\partial^2}{\partial z^2}\right)\left(@coeff(avect,i)@\right) \, \textbf{i} + \left(\frac{\partial^2}{\partial x^2} + \frac{\partial^2}{\partial y^2} + \frac{\partial^2}{\partial z^2}\right)\left(@coeff(avect,j)@\right) \, \textbf{j} + \left(\frac{\partial^2}{\partial x^2} + \frac{\partial^2}{\partial y^2} + \frac{\partial^2}{\partial z^2}\right)\left(@coeff(avect,k)@\right) \, \textbf{k} \end{align*}
\subsubsection{Model Solution}
(a) We have \begin{align*} \nabla \cdot \mathbf{a} &= \frac{\partial}{\partial x}\left(@coeff(avect,i)@\right) + \frac{\partial}{\partial y}\left(@coeff(avect,j)@\right) + \frac{\partial}{\partial z}\left(@coeff(avect,k)@\right) \\ &= \left(@diff(coeff(avect,i),x)@\right) + \left(@diff(coeff(avect,j),y)@\right) + \left(@diff(coeff(avect,k),z)@\right) \\ &= @ta1@ \end{align*}(b) We can use our answer from (a) to get \begin{align*} \nabla(\nabla \cdot \mathbf{a}) &= \frac{\partial}{\partial x} \left(@ta1@\right) \, \textbf{i} + \frac{\partial}{\partial y} \left(@ta1@\right) \,\textbf{j} + \frac{\partial}{\partial z}\left(@ta1@\right) \,\textbf{k} \\ &= \left(@coeff(ta2,i)@\right) \,\textbf{i} + \left(@coeff(ta2,j)@\right) \,\textbf{j} + \left(@coeff(ta2,k)@\right) \,\textbf{k} \end{align*}(c) We have \begin{align*} \nabla^2 \mathbf{a} &= \left(\frac{\partial^2}{\partial x^2} + \frac{\partial^2}{\partial y^2} + \frac{\partial^2}{\partial z^2}\right)\textbf{a} \\ &= \left(\frac{\partial^2}{\partial x^2} + \frac{\partial^2}{\partial y^2} + \frac{\partial^2}{\partial z^2}\right)\left(@coeff(avect,i)@\right) \,\textbf{i} + \left(\frac{\partial^2}{\partial x^2} + \frac{\partial^2}{\partial y^2} + \frac{\partial^2}{\partial z^2}\right)\left(@coeff(avect,j)@\right) \,\textbf{j} + \left(\frac{\partial^2}{\partial x^2} + \frac{\partial^2}{\partial y^2} + \frac{\partial^2}{\partial z^2}\right)\left(@coeff(avect,k)@\right) \, \textbf{k} \\ &= \left(\left(@diff(coeff(avect,i),x,2)@\right)+\left(@diff(coeff(avect,i),y,2)@\right)+\left(@diff(coeff(avect,i),z,2)@\right)\right) \, \textbf{i} + \left(\left(@diff(coeff(avect,j),x,2)@\right)+\left(@diff(coeff(avect,j),y,2)@\right)+\left(@diff(coeff(avect,j),z,2)@\right)\right) \, \textbf{j} + \left(\left(@diff(coeff(avect,k),x,2)@\right)+\left(@diff(coeff(avect,k),y,2)@\right)+\left(@diff(coeff(avect,k),z,2)@\right)\right) \, \textbf{k} \\ &= \left(@coeff(ta3,i)@\right) \, \textbf{i} +\left(@coeff(ta3,j)@\right) \, \textbf{j} +\left(@coeff(ta3,k)@\right) \, \textbf{k} \end{align*}(d) We have \begin{align*} \nabla \times \mathbf{a} &= \begin{vmatrix} \mathbf{i} & \mathbf{j} & \mathbf{k} \\ \frac{\partial}{\partial x} & \frac{\partial}{\partial y} & \frac{\partial}{\partial z} \\ @coeff(avect,i)@ & @coeff(avect,j)@ & @coeff(avect,k)@ \end{vmatrix} \\ &= \left(\frac{\partial}{\partial y}@coeff(avect,k)@-\frac{\partial}{\partial z}@coeff(avect,j)@\right) \, \textbf{i} - \left(\frac{\partial}{\partial x}@coeff(avect,k)@-\frac{\partial}{\partial z}@coeff(avect,i)@ \right) \, \textbf{j} + \left(\frac{\partial}{\partial x}@coeff(avect,j)@-\frac{\partial}{\partial y}@coeff(avect,i)@\right) \, \textbf{k} \\ &= \left(@diff(coeff(avect,k),y)@-@diff(coeff(avect,j),z)@\right) \, \textbf{i} - \left(@diff(coeff(avect,k),x)@-@diff(coeff(avect,i),z)@\right) \, \textbf{j} + \left(@diff(coeff(avect,j),x)@-@diff(coeff(avect,i),y)@\right) \, \textbf{k} \\ &= \left(@coeff(ta4,i)@\right) \, \textbf{i} + \left(@coeff(ta4,j)@ \right) \, \textbf{j} + \left(@coeff(ta4,k)@\right) \, \textbf{k} \end{align*}(e) A vector field is a gradient field if and only if the curl of it is equal to zero. So, there exists a function, \(\phi\), such that \(\mathbf{a} = \nabla \phi\) if and only if $\nabla \times \textbf{a} = 0$. From (d), we see this is @ta5@.
\subsubsection{Question Note}
Calculate $\nabla \cdot \mathbf{a}$, $\nabla(\nabla \cdot \mathbf{a})$, $\nabla^2 \mathbf{a}$, $\nabla \times \mathbf{a}$ and state whether $\mathbf{a}$ is a gradient field for \(\mathbf a = @avect@\).
\subsubsection{Feedback Variables}
\begin{lstlisting}
/* Feedback Variables */
ecf2: trigsimp(Grad(ans1));
\end{lstlisting}

\subsection{Fundamental Theorem of Vector Calculus}
\subsubsection{Question Variables}
\begin{lstlisting}
powerdisp:true;
Grad(f) := diff(f,x)*i + diff(f,y)*j + diff(f,z)*k;

/* Creating phi and vector field a */
FuncList:[sin,cos,exp,log];
Func:[rand(4)+1,0,0];
Func[2]:rand_with_prohib(1,4,[Func[1]]);
Func[3]:rand_with_prohib(1,4,[Func[1],Func[2]]);
phi: FuncList[Func[1]](x) + FuncList[Func[2]](y) + FuncList[Func[3]](z);
avect: Grad(phi);

/* Choosing the co-ordinates. Chosen such that phi of A or B is ALWAYS a fraction, never needing functions or square roots. Designed such that no values of x,y,z are the same in A and B. */
NiceCoords:[[0,pi,-pi,-pi/6,-3*pi/6,-5*pi/6, pi/6,3*pi/6,5*pi/6], [0,pi/3,-pi/3,pi/2,-pi/2,2*pi/3,-2*pi/3,-pi,pi], [0,log(2),log(3)], [1,e,e^2]];
Anumbx: rand(length(NiceCoords[Func[1]]))+1;
Anumby: rand(length(NiceCoords[Func[2]]))+1;
Anumbz: rand(length(NiceCoords[Func[3]]))+1;
Bnumbx: rand_with_prohib(1,length(NiceCoords[Func[1]]),[Anumbx]);
Bnumby: rand_with_prohib(1,length(NiceCoords[Func[2]]),[Anumby]);
Bnumbz: rand_with_prohib(1,length(NiceCoords[Func[3]]),[Anumbz]);
coordA: [NiceCoords[Func[1]][Anumbx], NiceCoords[Func[2]][Anumby], NiceCoords[Func[3]][Anumbz]];
coordB: [NiceCoords[Func[1]][Bnumbx], NiceCoords[Func[2]][Bnumby], NiceCoords[Func[3]][Bnumbz]];

/* Randomise whether A or B is chosen for part (b) */
numb: rand(2)+1;
givenAorB: [A,B];
givenAorBactual: [coordA,coordB];
wantAorB: [B,A];
pmintegral: ["+"," -"];     /* Choosing A or B makes integral plus or minused in working out */

/* Teacher Answers */
ta1: integrate(coeff(avect,i),x,coordA[1],coordB[1]) + integrate(coeff(avect,j),y,coordA[2],coordB[2]) + integrate(coeff(avect,k),z,coordA[3],coordB[3]);
soln2: [subst([x=coordA[1],y=coordA[2],z=coordA[3]],phi) + ta1,subst([x=coordB[1],y=coordB[2],z=coordB[3]],phi) - ta1];     /* Both solutions for (b) for each case */
ta2: soln2[numb];
\end{lstlisting}
\subsubsection{Question Text}
A vector field, \(\textbf{a}\), is given by \[ \textbf{a} = @coeff(avect,i)@ \, \textbf{i} + @coeff(avect,j)@ \, \textbf{j} + @coeff(avect,k)@ \, \textbf{k} \](a) Compute the line integral, \(I\), from position \(A=(@coordA[1]@,@coordA[2]@,@coordA[3]@)\) to position \(B=(@coordB[1]@,@coordB[2]@,@coordB[3]@)\) given by\[ I = \int_{A}^{B} \textbf{a} \cdot \text{d}\textbf{r} \] Hint: remember that \(\text{d}\textbf{r} = \text{d}x \, \textbf{i} + \text{d}y \, \textbf{j} + \text{d}z \, \textbf{k}\). If you choose to answer in decimal form, give your answer to at least two decimal places.

\(I = \) [[input:ans1]][[validation:ans1]][[feedback:prt1]]

(b) We find that there exists a function \(\phi\) such that \[ \textbf{a} = \nabla \phi \]Given the value of \(\phi(@givenAorB[numb]@)=@subst([x=givenAorBactual[numb][1],y=givenAorBactual[numb][2],z=givenAorBactual[numb][3]],phi)@\) and using your solution from part (a), state the value of \(\phi(@wantAorB[numb]@)\). If you choose to answer in decimal form, give your answer to at least two decimal places.

\(\phi(@wantAorB[numb]@) = \) [[input:ans2]][[validation:ans2]][[feedback:prt2]]
\subsubsection{Hint}
Remember that the fundamental theorem of vector calculus states \[ \int_{A}^{B} \nabla \phi \cdot \text{d}\textbf{r} = \phi(B) - \phi(A) \] You may wish to use this formula to solve part (b).
\subsubsection{Model Solution}
(a) Using our hint, we get \begin{align*} I &= \int_{A}^{B} \textbf{a} \cdot \text{d}\textbf{r} \\ &= \int_{A}^{B} @coeff(avect,i)@ \, \text{d}x + @coeff(avect,j)@ \, \text{d}y + @coeff(avect,k)@ \, \text{d}z \\ &= \int_{@coordA[1]@}^{@coordB[1]@} @coeff(avect,i)@ \, \text{d}x + \int_{@coordA[2]@}^{@coordB[2]@} @coeff(avect,j)@ \, \text{d}y + \int_{@coordA[3]@}^{@coordB[3]@} @coeff(avect,k)@ \, \text{d}z \\ &= \left[@integrate(coeff(avect,i),x)@\right]_{@coordA[1]@}^{@coordB[1]@} + \left[@integrate(coeff(avect,j),y)@\right]_{@coordA[2]@}^{@coordB[2]@} + \left[@integrate(coeff(avect,k),z)@\right]_{@coordA[3]@}^{@coordB[3]@} \\ &= \left[@integrate(coeff(avect,i),x,coordA[1],coordB[1])@\right] + \left[@integrate(coeff(avect,j),y,coordA[2],coordB[2])@\right] + \left[@integrate(coeff(avect,k),z,coordA[3],coordB[3])@\right] \\ &= @ta1@ \end{align*} (b) Since $\textbf{a} = \nabla \phi$ is a gradient field, we can apply the fundamental theorem of vector calculus \[ \int_{A}^{B} \textbf{a} \cdot \text{d}\textbf{r} = \int_{A}^{B} \nabla \phi \cdot \text{d}\textbf{r} = \phi(B) - \phi(A) \] Hence, using our value of $\int_{A}^{B} \textbf{a} \cdot \text{d}\textbf{r}$ from (a), and the value of $\phi(@givenAorB[numb]@)$ given, we get \begin{align*} \phi(@wantAorB[numb]@) &= \phi(@givenAorB[numb]@) @pmintegral[numb]@ \int_{A}^{B} \textbf{a} \cdot \text{d}\textbf{r} \\ &= \left(@subst([x=givenAorBactual[numb][1],y=givenAorBactual[numb][2],z=givenAorBactual[numb][3]],phi)@\right) @pmintegral[numb]@ \left(@ta1@\right) \\ &= @ta2@ \end{align*}
\subsubsection{Question Note}
For \(\mathbf{a} = @avect@\), find \(I = \int_{A}^{B} \textbf{a} \cdot \text{d}\textbf{r}\). For \(\mathbf{a} = \nabla \phi\), given \(\phi(@givenAorB[numb]@)\), find \(\phi(@wantAorB[numb]@)\).
\subsubsection{Feedback Variables}
\begin{lstlisting}
/* Feedback Variables */
ecfbothsolns2: [subst([x=coordA[1],y=coordA[2],z=coordA[3]],phi) + ans1,subst([x=coordB[1],y=coordB[2],z=coordB[3]],phi) - ans1];
ecf2: ecfbothsolns[numb];
\end{lstlisting}

\subsection{Calculate Curl and Line Integral}
\subsubsection{Question Variables}
\begin{lstlisting}
powerdisp:true;
Curl(f) := (diff(coeff(f,k),y)-diff(coeff(f,j),z))*i + (diff(coeff(f,i),z)-diff(coeff(f,k),x))*j + (diff(coeff(f,j),x)-diff(coeff(f,i),y))*k;

/* Defining vector field a. It takes one dimension using a random monomial in x,y,z, and it takes two dimensions with random addition of nonzero powers. These two mathematically force the final line integral to never be an addition of two zero integrals */
funcs: [(rand_with_prohib(-3,3,[0]))*x^(rand(2)+1)*y^(rand(2)+1)*z^(rand(2)+1), rand_with_prohib(-2,2,[0])*x^(rand(2)+1)+rand_with_prohib(-2,2,[0])*y^(rand(2)+1)+rand_with_prohib(-2,2,[0])*z^(rand(2)+1), rand_with_prohib(-1,1,[0])*x^(rand(2)+1)+rand_with_prohib(-1,1,[0])*y^(rand(2)+1)+rand_with_prohib(-1,1,[0])*z^(rand(2)+1)];
avectnumb1: rand(3)+1;
avectnumb2: rand_with_prohib(1,3,[avectnumb1]);
avectnumb3: rand_with_prohib(1,3,[avectnumb1,avectnumb2]);
avectcomps[1]: funcs[avectnumb1];
avectcomps[2]: funcs[avectnumb2];
avectcomps[3]: funcs[avectnumb3];
avect: avectcomps[1]*i+avectcomps[2]*j+avectcomps[3]*k;

/* Multiple case scenarios of which direction to traverse to define B */
A: [0,0,0];
AtoBnumb: rand(6)+1;
ones: [[1,0,0],[-1,0,0],[0,1,0],[0,-1,0],[0,0,1],[0,0,-1]];     /* Later used to define B */
AtoBijk: [i,i,j,j,k,k];     /* Used to display direction in integral dot product */
AtoBup: [1,-1,1,-1,1,-1];      /* Upper limit in integral */
AtoBlow: [0,0,0,0,0,0];      /* Lower limit in integral */
AtoBxyz: [x,x,y,y,z,z];      /* Used in lower limit to specify variable  */
AtoBdxyz: [dx,-dx,dy,-dy,dz,-dz];     /* Used as value of dr */
AtoBxyzopps: [[y,z],[y,z],[x,z],[x,z],[x,y],[x,y]];     /* Used to say these variables are fixed */
B: [A[1]+ones[AtoBnumb][1], A[2]+ones[AtoBnumb][2],A[3]+ones[AtoBnumb][3]];     /* Calculates B */
Binxyz: B[1]*x+B[2]*y+B[3]*z;     /* Used to say B has value x=(0/1) etc. */

/* Multiple case scenarios of which direction to traverse to define C */
BtoCnumb: rand_with_prohib(1,6,[AtoBnumb,AtoBnumb+1,AtoBnumb-1]);     /* Removes case of travelling back to A or another step in same direction */
ones: [[1,0,0],[-1,0,0],[0,1,0],[0,-1,0],[0,0,1],[0,0,-1]];     /* Later used to define C */
BtoCijk: [i,i,j,j,k,k];     /* Used to display direction in integral dot product */
BtoCup: [1,-1,1,-1,1,-1];      /* Upper limit in integral */
BtoClow: [0,0,0,0,0,0];      /* Lower limit in integral */
BtoCxyz: [x,x,y,y,z,z];      /* Used in lower limit to specify variable  */
BtoCdxyz: [dx,-dx,dy,-dy,dz,-dz];     /* Used as value of dr */
BtoCxyzopps: [[y,z],[y,z],[x,z],[x,z],[x,y],[x,y]];     /* Used to say these variables are fixed */
C: [B[1]+ones[BtoCnumb][1], B[2]+ones[BtoCnumb][2],B[3]+ones[BtoCnumb][3]];     /* Calculates C */

/* Teacher Answers */
ta1: Curl(avect);
anspart1: integrate(subst([AtoBxyzopps[AtoBnumb][1]=0,AtoBxyzopps[AtoBnumb][2]=0],coeff(avect,AtoBijk[AtoBnumb])*AtoBup[AtoBnumb]),AtoBxyz[AtoBnumb],AtoBlow[AtoBnumb],AtoBup[AtoBnumb]);     /* Calculates integral from A to B */
anspart2: integrate(subst([BtoCxyzopps[BtoCnumb][1]=coeff(Binxyz,BtoCxyzopps[BtoCnumb][1]),BtoCxyzopps[BtoCnumb][2]=coeff(Binxyz,BtoCxyzopps[BtoCnumb][2])],coeff(avect,BtoCijk[BtoCnumb])*BtoCup[BtoCnumb]),BtoCxyz[BtoCnumb],BtoClow[BtoCnumb],BtoCup[BtoCnumb]);     /* Calculates integral from B to C */
ta2: anspart1+anspart2;
\end{lstlisting}
\subsubsection{Question Text}
For a vector field, \(\textbf{a}\), given by \[ \textbf{a} = \left(@coeff(avect,i)@\right) \, \textbf{i} + \left(@coeff(avect,j)@\right) \, \textbf{j} + \left(@coeff(avect,k)@\right) \, \textbf{k} \](a) Find \(\nabla \times \textbf{a} \). Use the suggested form for your solution which has been provided.

\(\nabla \times \textbf{a} = \) [[input:ans1]][[validation:ans1]][[feedback:prt1]]

(b) For the vector field given by \(\textbf{a}\), consider a path from a position \(A=(@A[1]@,@A[2]@,@A[3]@)\) to position \(B=(@B[1]@,@B[2]@,@B[3]@)\), and then from \(B\) to position \(C=(@C[1]@,@C[2]@,@C[3]@)\). Evaluate the integral \[ I = \int_{ABC} \textbf{a} \cdot \text{d}\textbf{r} \]Hint: consider the two parts of the path separately, from \(A\) to \(B\), and then from \(B\) to \(C\). If you choose to answer in decimal form, give your answer to at least two decimal places.

\(I = \) [[input:ans2]][[validation:ans2]][[feedback:prt2]]
\subsubsection{Hint}
Try splitting the integral \(I\) up into two parts as \[ I = \int_{ABC} \textbf{a} \cdot \text{d}\textbf{r} = \int_{AB} \textbf{a} \cdot \text{d}\textbf{r} + \int_{BC} \textbf{a} \cdot \text{d}\textbf{r} \] You should be able to compute each integral individually by considering the straight line paths from \(A\) to \(B\), and \(B\) to \(C\), by using the appropriate vector for \(\text{d}\textbf{r}\) for each integral.
\subsubsection{Model Solution}
(a) We find the curl of \(\textbf{a}\) as \begin{align*} \nabla \times \mathbf{a} &= \begin{vmatrix} \mathbf{i} & \mathbf{j} & \mathbf{k} \\ \frac{\partial}{\partial x} & \frac{\partial}{\partial y} & \frac{\partial}{\partial z} \\ \left(@coeff(avect,i)@\right) & \left(@coeff(avect,j)@\right) & \left(@coeff(avect,k)@\right) \end{vmatrix} \\ &= \left(@diff(coeff(avect,k),y)@ - @diff(coeff(avect,j),z)@\right) \textbf{i} - \left(@diff(coeff(avect,k),x)@ - @diff(coeff(avect,i),z)@\right) \textbf{j} + \left(@diff(coeff(avect,j),x)@ - @diff(coeff(avect,i),y)@\right) \textbf{k} \\ &= \left(@diff(coeff(avect,k),y) - diff(coeff(avect,j),z)@\right) \textbf{i} + \left(@-diff(coeff(avect,k),x) + diff(coeff(avect,i),z)@\right) \textbf{j} + \left(@diff(coeff(avect,j),x) - diff(coeff(avect,i),y)@\right) \textbf{k} \end{align*} (b) We shall separate the integral \(I\) into the two parts suggested in the hint \[ I = \int_{ABC} \textbf{a} \cdot \text{d}\textbf{r} = \int_{AB} \textbf{a} \cdot \text{d}\textbf{r} + \int_{BC} \textbf{a} \cdot \text{d}\textbf{r} \] So, we compute the first integral \begin{align*} \int_{AB} \textbf{a} \cdot \text{d}\textbf{r} &= \int_{(@A[1]@,@A[2]@,@A[3]@)}^{(@B[1]@,@B[2]@,@B[3]@)} \textbf{a} \cdot \text{d}\textbf{r} \\ &= \int_{@AtoBxyz[AtoBnumb]@ = @AtoBlow[AtoBnumb]@}^{@AtoBup[AtoBnumb]@} \left(\left(@coeff(avect,i)@\right) \, \textbf{i} + \left(@coeff(avect,j)@\right) \, \textbf{j} + \left(@coeff(avect,k)@\right) \, \textbf{k} \right)\cdot @AtoBdxyz[AtoBnumb]@ \, \textbf{@AtoBijk[AtoBnumb]@} \\ &= \int_{@AtoBxyz[AtoBnumb]@ = @AtoBlow[AtoBnumb]@}^{@AtoBup[AtoBnumb]@} @coeff(avect,AtoBijk[AtoBnumb])*AtoBup[AtoBnumb]@ \, \text{d} @AtoBxyz[AtoBnumb]@ \end{align*} However, we know that this path keeps fixed values of \(@AtoBxyzopps[AtoBnumb][1]@=0\) and \(@AtoBxyzopps[AtoBnumb][2]@=0\). So, we can substitute these values into the integral \begin{align*} \int_{AB} \textbf{a} \cdot \text{d}\textbf{r} &= \int_{@AtoBxyz[AtoBnumb]@ = @AtoBlow[AtoBnumb]@}^{@AtoBup[AtoBnumb]@} @coeff(avect,AtoBijk[AtoBnumb])*AtoBup[AtoBnumb]@ \, \text{d} @AtoBxyz[AtoBnumb]@ \\ &= \int_{@AtoBxyz[AtoBnumb]@ = @AtoBlow[AtoBnumb]@}^{@AtoBup[AtoBnumb]@} @subst([AtoBxyzopps[AtoBnumb][1]=0,AtoBxyzopps[AtoBnumb][2]=0],coeff(avect,AtoBijk[AtoBnumb])*AtoBup[AtoBnumb])@ \, \text{d} @AtoBxyz[AtoBnumb]@ \\ &= \left[ @integrate(subst([AtoBxyzopps[AtoBnumb][1]=0,AtoBxyzopps[AtoBnumb][2]=0],coeff(avect,AtoBijk[AtoBnumb])*AtoBup[AtoBnumb]),AtoBxyz[AtoBnumb])@ \right]^{@AtoBup[AtoBnumb]@}_{@AtoBlow[AtoBnumb]@} \\ &= @integrate(subst([AtoBxyzopps[AtoBnumb][1]=0,AtoBxyzopps[AtoBnumb][2]=0],coeff(avect,AtoBijk[AtoBnumb])*AtoBup[AtoBnumb]),AtoBxyz[AtoBnumb],AtoBlow[AtoBnumb],AtoBup[AtoBnumb])@ \end{align*}Now, we compute the second integral similarly. \begin{align*} \int_{BC} \textbf{a} \cdot \text{d}\textbf{r} &= \int_{(@B[1]@,@B[2]@,@B[3]@)}^{(@C[1]@,@C[2]@,@C[3]@)} \textbf{a} \cdot \text{d}\textbf{r} \\ &= \int_{@BtoCxyz[BtoCnumb]@ = @BtoClow[BtoCnumb]@}^{@BtoCup[BtoCnumb]@} \left(\left(@coeff(avect,i)@\right) \, \textbf{i} + \left(@coeff(avect,j)@\right) \, \textbf{j} + \left(@coeff(avect,k)@\right) \, \textbf{k} \right)\cdot @AtoBdxyz[BtoCnumb]@ \, \textbf{@BtoCijk[BtoCnumb]@} \\ &= \int_{@BtoCxyz[BtoCnumb]@ = @BtoClow[BtoCnumb]@}^{@BtoCup[BtoCnumb]@} @coeff(avect,BtoCijk[BtoCnumb])*BtoCup[BtoCnumb]@ \, \text{d} @BtoCxyz[BtoCnumb]@ \end{align*}However, we know that this path keeps fixed values of \(@BtoCxyzopps[BtoCnumb][1]@=@coeff(Binxyz,BtoCxyzopps[BtoCnumb][1])@\) and \(@BtoCxyzopps[BtoCnumb][2]@=@coeff(Binxyz,BtoCxyzopps[BtoCnumb][2])@\). So, we can substitute these values into the integral \begin{align*} \int_{BC} \textbf{a} \cdot \text{d}\textbf{r} &= \int_{@BtoCxyz[BtoCnumb]@ = @BtoClow[BtoCnumb]@}^{@BtoCup[BtoCnumb]@} @coeff(avect,BtoCijk[BtoCnumb])*BtoCup[BtoCnumb]@ \, \text{d} @BtoCxyz[BtoCnumb]@ \\ &= \int_{@BtoCxyz[BtoCnumb]@ = @BtoClow[BtoCnumb]@}^{@BtoCup[BtoCnumb]@} @subst([BtoCxyzopps[BtoCnumb][1]=coeff(Binxyz,BtoCxyzopps[BtoCnumb][1]),BtoCxyzopps[BtoCnumb][2]=coeff(Binxyz,BtoCxyzopps[BtoCnumb][2])],coeff(avect,BtoCijk[BtoCnumb])*BtoCup[BtoCnumb])@ \, \text{d} @BtoCxyz[BtoCnumb]@ \\ &= \left[ @integrate(subst([BtoCxyzopps[BtoCnumb][1]=coeff(Binxyz,BtoCxyzopps[BtoCnumb][1]),BtoCxyzopps[BtoCnumb][2]=coeff(Binxyz,BtoCxyzopps[BtoCnumb][2])],coeff(avect,BtoCijk[BtoCnumb])*BtoCup[BtoCnumb]),BtoCxyz[BtoCnumb])@ \right]^{@BtoCup[BtoCnumb]@}_{@BtoClow[BtoCnumb]@} \\ &= @integrate(subst([BtoCxyzopps[BtoCnumb][1]=coeff(Binxyz,BtoCxyzopps[BtoCnumb][1]),BtoCxyzopps[BtoCnumb][2]=coeff(Binxyz,BtoCxyzopps[BtoCnumb][2])],coeff(avect,BtoCijk[BtoCnumb])*BtoCup[BtoCnumb]),BtoCxyz[BtoCnumb],BtoClow[BtoCnumb],BtoCup[BtoCnumb])@ \end{align*}Hence, we add the two parts to get our solution \begin{align*} I &= \int_{ABC} \textbf{a} \cdot \text{d}\textbf{r} \\ &= \int_{AB} \textbf{a} \cdot \text{d}\textbf{r} + \int_{BC} \textbf{a} \cdot \text{d}\textbf{r} \\ &= \left( @anspart1@ \right) + \left( @anspart2@ \right) \\ &= @ta2@ \end{align*}
\subsubsection{Question Note}
Find the curl of \(\mathbf{a}=@avect@\) and evaluate the integral \( I = \int_{ABC} \textbf{a} \cdot \text{d}\textbf{r} \) where \(A=@A@, B=@B@, C=@C@\).

\subsection{More Differential Operators}
\subsubsection{Question Variables}
\begin{lstlisting}
powerdisp:true;

/* Defining vector field A. It chooses 3 different functions from 'funcs' */
funcs: [cos(rand([x,y,z])),sin(rand([x,y,z])),exp(rand([x,y,z])),log(rand([x,y,z])),rand_with_prohib(-5,5,[0])*x^(rand(3)+1),rand_with_prohib(-5,5,[0])*y^(rand(3)+1),rand_with_prohib(-5,5,[0])*z^(rand(3)+1)];
icompnumb: rand(7)+1;
jcompnumb: rand_with_prohib(1,7,[icompnumb]);
kcompnumb: rand_with_prohib(1,7,[icompnumb,jcompnumb]);
aveccomps: [funcs[icompnumb],funcs[jcompnumb],funcs[kcompnumb]];
avec: aveccomps[1]*i+aveccomps[2]*j+aveccomps[3]*k;

/* Defining vector field B. It chooses 3 different functions from 'funcs2' */
funcs2: [cos(rand([x,y,z])),sin(rand([x,y,z])),exp(rand([x,y,z])),log(rand([x,y,z])),rand_with_prohib(-5,5,[0])*x^(rand(3)+1),rand_with_prohib(-5,5,[0])*y^(rand(3)+1),rand_with_prohib(-5,5,[0])*z^(rand(3)+1)];     /* A replica of funcs is used to randomly generate the coefficients and arguments of each function, to avoid repetition from A */
icompnumb2: rand(7)+1;
jcompnumb2: rand_with_prohib(1,7,[icompnumb2]);
kcompnumb2: rand_with_prohib(1,7,[icompnumb2,jcompnumb2]);
bveccomps: [funcs2[icompnumb2],funcs2[jcompnumb2],funcs2[kcompnumb2]];
bvec: bveccomps[1]*i+bveccomps[2]*j+bveccomps[3]*k;

/* Randomly generate which term we find coeff of */
ijk: rand([i,j,k]);

/* Teacher Answer */
ta1: trigsimp(coeff(avec,i)*diff(coeff(bvec,ijk),x)+coeff(avec,j)*diff(coeff(bvec,ijk),y)+coeff(avec,k)*diff(coeff(bvec,ijk),z));     /* By design of question and functions, answer is never 0 */
\end{lstlisting}
\subsubsection{Question Text}
Consider the two vector fields \begin{align*} \mathbf A &= @coeff(avec,i)@ \, \mathbf i + @coeff(avec,j)@ \, \mathbf j + @coeff(avec,k)@ \, \mathbf k \\ \mathbf B &= @coeff(bvec,i)@ \, \mathbf i + @coeff(bvec,j)@ \, \mathbf j + @coeff(bvec,k)@ \, \mathbf k \end{align*}Find the coefficient of the \(\mathbf{@ijk@}\) term in the expression \( (\mathbf A\cdot\nabla)\mathbf B \). Note that \(e^{x}\) and \(\ln(x)\) can be typed as exp(x), and log(x) or ln(x) in the answer box.

[[input:ans1]][[validation:ans1]][[feedback:prt1]]
\subsubsection{Hint}
Note that \begin{align*} \textbf{A} \cdot \nabla \neq \nabla \cdot \textbf{A} \end{align*} The right-hand side is the divergence of \(\textbf{A}\), whereas the left-hand side is something quite different. You can find it by taking the dot product of \(\textbf{A}\) with \(\nabla\), where \(\nabla = \frac{\partial}{\partial x} \, \textbf{i} + \frac{\partial}{\partial y} \, \textbf{j} + \frac{\partial}{\partial z} \, \textbf{k}\).
\subsubsection{Model Solution}
Firstly, we should note that \begin{align*} \textbf{A} \cdot \nabla \neq \nabla \cdot \textbf{A} \end{align*} The right-hand side is the divergence of $\textbf{A}$, whereas the left-hand side is something quite different. By taking the dot product of \(\textbf{A}\) with \(\nabla\), where  \(\nabla = \frac{\partial}{\partial x} \, \textbf{i} + \frac{\partial}{\partial y} \, \textbf{j} + \frac{\partial}{\partial z} \, \textbf{k}\), we see that it is an operator \begin{align*} \textbf{A} \cdot \nabla &= @coeff(avec,i)@ \frac{\partial}{\partial x} +@coeff(avec,j)@ \frac{\partial}{\partial y} +@coeff(avec,k)@ \frac{\partial}{\partial z} \\ \end{align*}Hence, \begin{align*} \left(\textbf{A} \cdot \nabla\right) \textbf{B} &= \left(@coeff(avec,i)@ \frac{\partial}{\partial x} +@coeff(avec,j)@ \frac{\partial}{\partial y} +@coeff(avec,k)@ \frac{\partial}{\partial z}\right) \left( @coeff(bvec,i)@ \, \textbf{i} + @coeff(bvec,j)@ \, \textbf{j} + @coeff(bvec,k)@ \, \textbf{k} \right) \end{align*} So, the coefficient of the $\textbf{@ijk@}$ term in $\left(\textbf{A} \cdot \nabla\right) \textbf{B}$ is given by \begin{align*} @coeff(avec,i)@ \frac{\partial}{\partial x}\left(@coeff(bvec,ijk)@\right) +@coeff(avec,j)@ \frac{\partial}{\partial y}\left(@coeff(bvec,ijk)@\right) +@coeff(avec,k)@ \frac{\partial}{\partial z}\left(@coeff(bvec,ijk)@\right) &= @coeff(avec,i)@ \cdot @diff(coeff(bvec,ijk),x)@ +@coeff(avec,j)@ \cdot @diff(coeff(bvec,ijk),y)@ +@coeff(avec,k)@ \cdot @diff(coeff(bvec,ijk),z)@ \\ &= @ta1@ \end{align*}
\subsubsection{Question Note}
Find the coefficient of the \(\mathbf{@ijk@}\) component of \((\mathbf A\cdot\nabla)\mathbf B \) for \(\mathbf A = @avec@\) and \(\mathbf B = @bvec@\).

\end{document}
